\documentclass[hidelinks, 12pt]{article}

\makeatletter
\setlength{\@fptop}{0pt}
\makeatother

\makeatletter

\newcommand\footnoteref[1]{\protected@xdef\@thefnmark{\ref{#1}}\@footnotemark}
\makeatother
\usepackage{graphicx}
\usepackage{graphicx,hyperref,amsmath,bm,url}
\usepackage[numbers]{natbib}
\usepackage{microtype,todonotes}
\usepackage{a4}
\usepackage[compact,small]{titlesec}
\usepackage[utf8]{inputenc}
\usepackage{placeins}
\clubpenalty = 10000
\widowpenalty = 10000
\usepackage[T1]{fontenc}
\graphicspath{ {../Billeder/} }
\usepackage{tikz}
\usetikzlibrary{calc}
\usetikzlibrary{shapes}
\usepackage[labelfont=bf]{caption}
\renewcommand{\figurename}{\textbf{Figur}}
\renewcommand{\contentsname}{Indholdfortegnelse}
\usepackage[nottoc,notlof,notlot]{tocbibind} 
\renewcommand\bibname{Referencer}
\renewcommand{\tablename}{Tabel}


\makeatletter
\newdimen\@myBoxHeight%
\newdimen\@myBoxDepth%
\newdimen\@myBoxWidth%
\newdimen\@myBoxSize%
\newcommand{\SquareBox}[2][]{%
    \settoheight{\@myBoxHeight}{#2}% Record height of box
    \settodepth{\@myBoxDepth}{#2}% Record depth of box
    \settowidth{\@myBoxWidth}{#2}% Record width of box
    \pgfmathsetlength{\@myBoxSize}{max(\@myBoxWidth,(\@myBoxHeight+\@myBoxDepth))}%
    \tikz \node [shape=rectangle, shape aspect=1,draw=red,inner sep=2\pgflinewidth, minimum size=\@myBoxSize,#1] {#2};%
}%
\makeatother
\newcommand*{\captionsource}[2]{%
  \caption[{#1}]{%
    #1%
    \\\hspace{\linewidth}%
    \textbf{Kilde:} #2%
  }%
}
\newcommand{\tabitem}{~~\llap{\textbullet}~~}
\begin{document}
	
	\title{\LaTeX\, minikursus}
	\author{Mathias Jakobsen}
	\date{\today}
	\maketitle %Sæt infoen ovenfra ind i dokumentet
	
	\section{Overskrift}
	\label{label_til_at_kende_afsnittet}
	Denne tekst er en del af afsnit \ref{label_til_at_kende_afsnittet}. %\ref indsætter afsnittets tal.
	\subsection{Underoverskrift}
	\label{endnu_en_label_til_dette_afsnit}
	Denne tekst er en del af underafsnittet \ref{endnu_en_label_til_dette_afsnit}.
	\subsubsection{Under-underoverskrift}
	You get the point.
	\begin{itemize}
		\item Sådan
		\item laver
		\item man
		\item en
		\item liste
	\end{itemize}
	
	Vi kan skrive med \textit{kursiv} her mellem listerne!

	Dog hvis vi vil skrive flotte citater foregår det med disse ``flotte tegn'' frem for de normale kedelige citationstegn.

	\begin{enumerate}
		\item Sådan
		\item laver
		\item man
		\item en
		\item numereret
		\item liste
	\end{enumerate}

	\begin{description}
		\item[Sidst] men ikke mindst har vi en liste der er god til begrebsforklaringer. Wuhuu, den er vi glad for.
		\item[Begrebsforklaring] er en meget smuk ting, og vi er meget glade for at vi har muligheden for at gøre det i \LaTeX.
	\end{description}

	\section*{Det skønne matematik} % "*" fjerner numereringen i dokumentet.
	Matematiske udtryk skrives således: $x = 4$, hvis de skal have deres egen linje er det $$ x = 4 $$. Hvis udtrykket skal nummereres bliver det:
	\begin{equation}
		c = \int^{a}_{b} x^2  dx
	\end{equation}

	\begin{figure}[t!]
    	\centering
    	\includegraphics[width=0.3\textwidth]{../Billeder/aaulogo}
    	\captionsource{Billedets titel}{\url{http://www.kilde.til.billede.com}}
    	\label{label til billede}
	\end{figure}

	Refference til en bog \cite{bog}

	\bibliographystyle{plain}
	\bibliography{../Rapport/sources}
\end{document}