	\documentclass[hidelinks, 12pt]{article}
\makeatletter
\newcommand{\group}{B2-2}
\newcommand\footnoteref[1]{\protected@xdef\@thefnmark{\ref{#1}}\@footnotemark}
\makeatother
\usepackage{graphicx}
\usepackage{graphicx,hyperref,amsmath,bm,url}
\usepackage[numbers]{natbib}
\usepackage{microtype,todonotes}
\usepackage{a4}
\usepackage[compact,small]{titlesec}
\usepackage[utf8]{inputenc}
\clubpenalty = 10000
\widowpenalty = 10000
\usepackage[T1]{fontenc}
\graphicspath{ {../Billeder/} }
\usepackage{tikz}
\usetikzlibrary{calc}
\usetikzlibrary{shapes}
\usepackage[labelfont=bf]{caption}
\renewcommand{\figurename}{\textbf{Figur}}
\renewcommand{\contentsname}{Indholdfortegnelse}
\renewcommand{\refname}{Referencer}

\makeatletter
\newdimen\@myBoxHeight%
\newdimen\@myBoxDepth%
\newdimen\@myBoxWidth%
\newdimen\@myBoxSize%
\newcommand{\SquareBox}[2][]{%
    \settoheight{\@myBoxHeight}{#2}% Record height of box
    \settodepth{\@myBoxDepth}{#2}% Record depth of box
    \settowidth{\@myBoxWidth}{#2}% Record width of box
    \pgfmathsetlength{\@myBoxSize}{max(\@myBoxWidth,(\@myBoxHeight+\@myBoxDepth))}%
    \tikz \node [shape=rectangle, shape aspect=1,draw=red,inner sep=2\pgflinewidth, minimum size=\@myBoxSize,#1] {#2};%
}%
\makeatother
\newcommand*{\captionsource}[2]{%
  \caption[{#1}]{%
    #1%
    \\\hspace{\linewidth}%
    \textbf{Kilde:} #2%
  }%
}
\newcommand{\tabitem}{~~\llap{\textbullet}~~}
\begin{document}
	
	\title{Samarbejdsaftale}
	\author{Gruppe \group}
	\date{\today \\\vspace{0.5cm} Version 1}
	\maketitle
	
	\section*{Beskrivelse af gruppe \group}
	Gruppen er bestående af følgende medlemmer, med følgende karakteristika inden for deres indlæring:

	\begin{table}[h!]
	\centering
	\begin{tabular}{|l|l|}
	\hline
	\textbf{Navn}            & \textbf{Karakteristika}                                                                             \\ \hline
	Mathias Steen Jakobsen   & \begin{tabular}[c]{@{}l@{}} \tabitem Reflektiv \textbf{(1)}\\  \tabitem Intuitiv \textbf{(1)}\\  \tabitem Verbal \textbf{(7)}\\  \tabitem Sekventiel \textbf{(7)}\end{tabular}  \\ \hline
	Frederik Lund Kær             & \begin{tabular}[c]{@{}l@{}} \tabitem Aktiv \textbf{(7)}\\  \tabitem Sansende \textbf{(7)}\\  \tabitem Visuel \textbf{(7)}\\  \tabitem Global \textbf{(3)}\end{tabular}  \\ \hline
	Jacob Askløf Svenningsen & \begin{tabular}[c]{@{}l@{}} \tabitem Reflektiv \textbf{(1)}\\  \tabitem Sansende \textbf{(7)}\\  \tabitem Visuel \textbf{(7)}\\  \tabitem Sekventiel \textbf{(3)}\end{tabular}  \\ \hline
	Morten Hansen            & \begin{tabular}[c]{@{}l@{}} \tabitem Reflektiv \textbf{(1)}\\  \tabitem Sansende \textbf{(9)}\\  \tabitem Visuel \textbf{(11)}\\  \tabitem Sekventiel \textbf{(3)}\end{tabular} \\ \hline
	Søren Madsen             & \begin{tabular}[c]{@{}l@{}} \tabitem Aktiv \textbf{(3)}\\  \tabitem Sansende \textbf{(7)}\\  \tabitem Verbal \textbf{(1)}\\  \tabitem Sekventiel \textbf{(5)}\end{tabular}    \\ \hline
	Theresa Krogh Walker           & \begin{tabular}[c]{@{}l@{}} \tabitem Reflektiv \textbf{(7)}\\  \tabitem Sansende \textbf{(9)}\\ \tabitem Visuel \textbf{(3)}\\  \tabitem Sekventiel \textbf{(1)}\end{tabular} \\ \hline
	\end{tabular}
	\end{table}

	Vi kan se i skemaet at alle på nær \'et gruppemedlem er sekventielle. Det betyder at vi ikke har brug for et stort overblik for at forstå vores emner, samt at vi har nemt ved at forklare vores viden. I projektet vil vi bruge dette, i form af gruppedelingsmøder. Vores gruppedelingsmøder er fastaftalte tidspunkter, for hver person gennemgår den viden personen har opnået siden sidste møde. Dermed sikres at alle i gruppen lærer alt omkring projektet, og at alle er sat lige meget ind i stoffet. Det giver også grobund for at kunne læse og rette hinandens delafsnit i rapporten.

	Vi er i gruppen en blanding af verbale og visuelle mennesker. Det vil sige at vi skal sørge for at bruge grafer og illustrationer som et redskab til at \textbf{understrege} vores pointer i det viden vi deler. Vi skal dog samtidig sørge for at kunne forklare alt vores viden, så man ikke kun afhænger af illustrationen.

	At vi er mange sansende gruppemedlemmer betyder, at når vi udveksler viden skal vi sørge for at bruge eksempler for at alle mennesker får noget ud af videndelingen. Når vi f.eks. sidder og programmerer i fællesskab er det også vigtigt at gå igennem eksempler på brugen af et kodestykke, så der kommer faktiske tal på bordet, i stedet for de mere abstrakte variable. 

	Vi er mange refleksive mennesker i gruppen, hvilket betyder at vi holder af at arbejde enten alene eller i meget små grupper. I vores projekt kunne vi derfor med fordel bruge tiden i grupperummene på at arbejde enten selv eller parvis og bruge de andre gruppemedlemmer til sparring, hvis man løber ind i et problem.

	\section*{Konflikthåndtering}
	Hvis der er et problem med et enkelt gruppemedlem, vil \'et andet gruppemedlem snakke på tomandshånd med personen, så der ikke er nogen der føler sig ``angrebet'', ved at hele gruppen fortæller om problemet på samme tid. 

	Vi skal sørge for at formulere os konstruktivt over for de personer vi ønsker ændrer adfærd. Konflikthåndteringen må ikke blive for ``stift'' og ``formelt'', det skal ikke virke planlagt, men mere som en normal samtale mellem mennesker. Dermed vil vi sikre os at folk ikke sidder og føler at folk gør klar til at hakke ned på dem inden et evt. fastlagt tidspunkt til konflikthåndtering.

	Hvis folk begynder ikke at overholde de aftaler der bliver lavet i gruppen, skal et andet gruppemedlem først snakke med personen om hvad årsagen til dette er. Hvis ikke der er en simpel løsning på problemet, eller hvis det er et problem omkring gruppen, tages dette op på et gruppemøde i stedet. Hvis ikke en løsning findes her, tages det videre til vejlederen, dette er dog først i ydertilfælde at dette skulle blive nødvendigt.

	Vi skal i gruppen sørge for at holde resten af gruppen informeret, hvis der opstår nogle problemer for et gruppemedlem (som f.eks. at man ikke kan møde op). Dermed vil vi sikre os at der ikke opstår konflikter grundet misinformering.

	\section*{Aftaler i gruppen}
	Første udgave af disse regler er opstillet d. 06/10-2015, og beskriver hvilke regler der gælder for gruppe \group, i forbindelse med P1-forløbet.

	\begin{itemize}
		\item Rettelser eller tilføjelser til gruppeaftalen, kan kun forekomme hvis der er et flertal af gruppemedlemmerne der er enige om ændringen. Det vil sige at i gruppen skal mindst 4 mennesker være til stede og enige om ændringen af samarbejdsaftalen.
		\item Vi skal sørge for at overholde de aftaler der besluttes på gruppen. Dette indebærer mundtlige både mundtlige og skriftlige aftaler. F.eks. i form af beslutninger på møder eller den skrevne tidsplan.
		\item Vi skal sørge for at være studieaktive. Det vil sige at vi skal møde op til forelæsninger og opgaveregning, så vi i gruppen kan afholde vigtige diskussioner, og hjælpe hele gruppen til en bedre indlæring. Hvis man ikke kan møde op til en forelæsning grundet en gyldig årsag (eks. sygdom), er man selv ansvarlig for at læse op på stoffet så man kan indgå i diskussioner om dette. Det vil sige at de andre gruppemedlemmer \textbf{ikke} er ansvarlige for at tage noter til personen.
		\item Ved gentagende brud på aftaler lavet i gruppen, tages det op på et gruppemøde. Hvis ikke der sker en forbedring af situationen inden for et par uger, tages det igen op på et gruppemøde, men hvor der denne gang skal diskuteres hvilke konsekvenser disse brud skal have. Ved grove overtrædelser, som ikke bliver ændret i tidsrummet mellem møderne, kan der blive stemt om at smide gruppemedlemmet ud af gruppen. I et sådan tilfælde skal alle gruppemedlemmer, på nær personen som har overtrådt reglerne, være enige om at medlemmet skal forlade gruppen.
		\item Når der er længere dage i grupperummet, afsættes der tid i slutningen af dagen til at lave videndeling, hvor gruppemedlemmerne hver især fremlægger hvad de har lært og skrevet om siden sidste gruppemøde.
		\item Det sidste punkt på hvert gruppemøde er at udarbejde en dagsorden for det næste møde. Denne dagsorden skal skrives ned, og deles mellem alle gruppemedlemmer ved hjælp af Git.
		\item Mødetiden i grupperummet er 8.15, medmindre der er en morgenforelæsning eller at andet er aftalt på et tidligere gruppemøde.
		\item Alle kilder der bruges i rapporten, skrives med det samme ind i filen ``sources.bib'', hvor der også skrives en kommentar til kilden, der beskriver hvad den handler om. 
		\item Når man har skrevet et afsnit gives dette videre til et andet gruppemedlem, der vil læse det igennem og rette det. Nogle gange i løbet af projektet vil vi aftale møder hvor hele gruppen i fællesskab går igennem rapporten og sikrer rød tråd og at sproget er konsekvent gennem hele rapporten.
		\item Når gruppen aftaler at holde en pause, skal hver person være indforstået med hvornår arbejdet skal forsættes, så pauserne ikke ender med at blive for lange og uhåndterbare.
		\item To dage før der skal afholdes vejledermøde besluttes på et gruppemøde en dagsorden. På denne dagsorden skrives der også en referent på dagsordenen. Denne rolle går på tur, og personen er forpligtet til at notere vigtige beslutninger der træffes på mødet.    
	\end{itemize}

\end{document}