\documentclass{article}
\usepackage[utf8]{inputenc}

\begin{document}
\title{Aflevering til statusseminar}
\author{Gruppe B2-2}
\date{\today}
\maketitle

\section*{Sendte filer}
Følgende filer er vedhæftet:
\begin{description}
	\item[B2-2\textunderscore Rapport.pdf] som er vores foreløbige rapport, komplet med bilag og refferencer. Til denne er der en detaljeret læsevejledning nedenfor
	\item[B2-2\textunderscore Gruppekontrakt.pdf] som indeholder vores samarbejdsaftale, samt resultater af Belbin test af gruppen
\end{description}

\section*{Læsevejledning til B2-2\textunderscore Rapport.pdf}
Vi ønsker at opponentgruppen er særlig opmærksom på følgende afsnit:
\begin{description}
	\item[2.4 - Kærbyskolen] som er en beskrivelse af den case vi har arbejdet med i projektet, samt en beskrivelse af hvordan denne skole lægger skemaer til dens elever.
	\item[2.5 - Interessenter] som er en beskrivelse af de forskellige interessenter som har en interesse i skemalægningen på folkeskoler.
	\item[2.6 - Pædagogisk læring] hvor forskellige synspunkter på pædagogisk læring fremføres, ud fra interviews vi har foretaget.
	\item[2.9 - SWOT] som er en SWOT-analyse af Kærbyskolen som virksomhed med fokus på skemalægningen. Vi opsummerer her vigtige opdagelser igennem projektet, samt ridser op hvilke udfordringer skemalægningen indebærer.
	\item[3 - Projektafgrænsning] hvor vi afgrænser hvilke dele af projektet vi vil arbejde videre med.
	\item[4 - Problemformulering] hvor vi koger afgrænsningen ned til et enkelt problem vi vil fokusere på.
	\item[5.1 - Løsningsstrategi] hvor vi beskriver hvilken tilgang vi vil tage til problemet, samt hvordan vi vil inddrage software i løsningen.
\end{description} 

\end{document}