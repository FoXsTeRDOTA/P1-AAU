\documentclass[hidelinks, 12pt]{article}



\makeatletter
\newcommand{\group}{B2-2}
\newcommand\footnoteref[1]{\protected@xdef\@thefnmark{\ref{#1}}\@footnotemark}
\makeatother
\usepackage{graphicx}
\usepackage{graphicx,hyperref,amsmath,bm,url}
\usepackage[numbers]{natbib}
\usepackage{microtype,todonotes}
\usepackage{a4}
\usepackage[compact,small]{titlesec}
\usepackage[utf8]{inputenc}
\usepackage{placeins}
\clubpenalty = 10000
\widowpenalty = 10000
\usepackage[T1]{fontenc}
\graphicspath{ {../Billeder/} }
\usepackage{tikz}
\usetikzlibrary{calc}
\usetikzlibrary{shapes}
\usepackage[labelfont=bf]{caption}
\renewcommand{\figurename}{\textbf{Figur}}
\renewcommand{\contentsname}{Indholdfortegnelse}
\usepackage[nottoc,notlof,notlot]{tocbibind} 
\renewcommand\bibname{Referencer}
\renewcommand{\tablename}{Tabel}


\makeatletter
\newdimen\@myBoxHeight%
\newdimen\@myBoxDepth%
\newdimen\@myBoxWidth%
\newdimen\@myBoxSize%
\newcommand{\SquareBox}[2][]{%
    \settoheight{\@myBoxHeight}{#2}% Record height of box
    \settodepth{\@myBoxDepth}{#2}% Record depth of box
    \settowidth{\@myBoxWidth}{#2}% Record width of box
    \pgfmathsetlength{\@myBoxSize}{max(\@myBoxWidth,(\@myBoxHeight+\@myBoxDepth))}%
    \tikz \node [shape=rectangle, shape aspect=1,draw=red,inner sep=2\pgflinewidth, minimum size=\@myBoxSize,#1] {#2};%
}%
\makeatother
\newcommand*{\captionsource}[2]{%
  \caption[{#1}]{%
    #1%
    \\\hspace{\linewidth}%
    \textbf{Kilde:} #2%
  }%
}
\newcommand{\tabitem}{~~\llap{\textbullet}~~}
\begin{document}
	
	\title{Dagsorden}
	\author{Gruppe \group}
	\date{14. oktober 2015}
	\maketitle
	
	Vi har kontaktet Sønderbro Skole pr mail, og spurgt om vi kan få aftalt et tidspunkt til at snakke med en skemalægger. Da der er efterårsferie i folkeskolerne, i denne uge, forventer vi dog ikke at modtage et svar inden mandag d. 19/10, når det er normal skoledag igen.  Vi har udarbejdet en række spørgsmål til skemalæggere og skolelærere, som vi ønsker at du læser igennem og kommer med feedback på. Spørgsmålene kan ses i bilag 2.

	Vi har opstillet en foreløbig plan for hvilke emner vores rapport skal indeholde. Vi ønsker feedback på denne, især i forhold til problemanalysen, og om vi undersøger emnet bredt nok. Denne plan kan ses i bilag 1.

	\subsection*{Bilag 1}
	\begin{itemize}
		\item Forside
		\item Titelblad
		\begin{itemize}
			\item Synopsis
		\end{itemize}
		\item Forord
		\item Indholdfortegnelse
		\begin{itemize}
			\item Motivation til at arbejde med emnet
			\item Initialiserende problem 
		\end{itemize}
		\item Problemanalyse
		\begin{itemize}
			\item Beskrivelse af skolereformen og dens betydning
			\item Lovgivning og regler
			\item Interessenter og andre på markedet
			\item Beskrivelse af cases og interviews med både skemalæggere, lærere og elever.
			\item Brugerovervejelser (hvilke kompetencer har brugerne af systemet)
			\item State of the art inden for skemalægningsværktøjer
		\end{itemize}
		\item Afgrænsning
		\item Problemformulering
		\item Problemløsning
		\begin{itemize}
			\item Teori (Gentiske algoritmer)
			\item Faktorer programmet skal tage højde for
			\item Udvikling og beskrivelse af programmet
		\end{itemize}
		\item Konklusion
		\item Kildeliste
		\item Bilag
	\end{itemize}

	\subsection*{Bilag 2}
	\subsubsection*{Spørgsmål til skemalægger}
	\begin{itemize}
		\item Kontaktoplysninger (hvis de giver lov)
		\item Oplysninger om skolen
		\begin{itemize}
			\item Størrelse
			\item Klasser
			\item Lærere
			\item Elever
			\item Lokaler
			\item Speciallokaer
			\begin{itemize}
				\item Fysik/kemi
				\item Idrætsområde
				\item Billedkunst
				\item Sløjd
				\item Musik
				\item Hjemkundskab
				\item Håndarbejde
			\end{itemize}
			\item Hvilke valgfag har I?
		\end{itemize}
		\item Hvad er jeres holdning til jeres skoles nuværende procedure angående skemalægning?
		\item Hvor lang tid bruger I på at lave skemaer? (før og efter reform?)
		\item Bruger skolen faste skemaer eller ugentlige?
		\item Hvor tit skal der ændres i skemaer?
		\item Er der faste lærer til hver klasse? Hvor længe beholder man samme lærer?
		\item Komfortabel med at arbejde i CLI?
		\item Har I faste lokaler?
		\item Hvordan håndterer i lokaler? (f.eks. ved store klasser?, hvad hvis 2 skal være i fysik-lokalet?)
		\item Hvordan håndterer I lærernes tid på skolen? (i form af forberedelsestid)
		\item Hvor mange timer om året pr lærer? pr elev? (Har skolen data?)
		\item Hvilket system bruger I til håndtering eller udarbejdning af skemaer?
		\item Hvordan samarbejder det med andre systemer? Kan du importere andre filer?
		\item Fremvisning af programmet?
		\item Prioritering af fag efter tid på dagen?
		\item Prioritering af læreres ønsker?
		\item Vikarhåndtering, bliver det sat ind i systemet?
		\item Bliver en vikartime regnet som en normal time? (uddannelsesniveauet er jo ikke det samme)
		\item Findes buffer-timer? 
		\item Hvordan håndterer man timer der ikke finder sted?
		\item Er lektiecafe på skemaet? Skal der være en lærer til stede (en lærer pr klasse)?
	\end{itemize}
	\subsubsection*{Spørgsmål til lærer}
	\begin{itemize}
	\item Hvordan ønsker du at tiden på skolen bruges (hvornår forberedelse og hvornår undervisning)
	\item Hvis vi laver systemet, hvilke parametre synes I så at vi skal tage højde for?
	\item Kopier af skemaer?
	\item Kopier af skemaer for elever?
	\item Hvor mange klasser er du lærer for?
	\item Hvordan mener du at skemalægningen kan forbedres?
	\end{itemize}


\end{document}