\documentclass[hidelinks, 12pt]{article}



\makeatletter
\newcommand{\group}{B2-2}
\newcommand\footnoteref[1]{\protected@xdef\@thefnmark{\ref{#1}}\@footnotemark}
\makeatother
\usepackage{graphicx}
\usepackage{graphicx,hyperref,amsmath,bm,url}
\usepackage[numbers]{natbib}
\usepackage{microtype,todonotes}
\usepackage{a4}
\usepackage[compact,small]{titlesec}
\usepackage[utf8]{inputenc}
\usepackage{placeins}
\clubpenalty = 10000
\widowpenalty = 10000
\usepackage[T1]{fontenc}
\graphicspath{ {../Billeder/} }
\usepackage{tikz}
\usetikzlibrary{calc}
\usetikzlibrary{shapes}
\usepackage[labelfont=bf]{caption}
\renewcommand{\figurename}{\textbf{Figur}}
\renewcommand{\contentsname}{Indholdfortegnelse}
\usepackage[nottoc,notlof,notlot]{tocbibind} 
\renewcommand\bibname{Referencer}
\renewcommand{\tablename}{Tabel}


\makeatletter
\newdimen\@myBoxHeight%
\newdimen\@myBoxDepth%
\newdimen\@myBoxWidth%
\newdimen\@myBoxSize%
\newcommand{\SquareBox}[2][]{%
    \settoheight{\@myBoxHeight}{#2}% Record height of box
    \settodepth{\@myBoxDepth}{#2}% Record depth of box
    \settowidth{\@myBoxWidth}{#2}% Record width of box
    \pgfmathsetlength{\@myBoxSize}{max(\@myBoxWidth,(\@myBoxHeight+\@myBoxDepth))}%
    \tikz \node [shape=rectangle, shape aspect=1,draw=red,inner sep=2\pgflinewidth, minimum size=\@myBoxSize,#1] {#2};%
}%
\makeatother
\newcommand*{\captionsource}[2]{%
  \caption[{#1}]{%
    #1%
    \\\hspace{\linewidth}%
    \textbf{Kilde:} #2%
  }%
}
\newcommand{\tabitem}{~~\llap{\textbullet}~~}
\begin{document}
	
	\title{Dagsorden}
	\author{Gruppe \group}
	\date{20. oktober 2015}
	\maketitle
	
	\section*{Spørgsmål til vejlederen}
	\begin{itemize}
		\item Hvordan skal vi håndtere interviewet i vores analyse, for at arbejde hen mod en løsning?
		\item Hvordan arbejder vi videre med at Sønderbro Skole ikke svarer / ikke har lyst til at samarbejde?
	\end{itemize}

	\section*{Læsevejledning}
	I den foreløbige rapport er der ikke skrevet mange afsnit, da vi har været lidt låste af ikke at have kunnet snakke med skolen. I bilaget kan ses et interview vi lavede med en sekretær på en skole (ikke Sønderbro skole). Sekretæren var ikke en del af det team der lagde skemaer, men vi tænkte at vi kunne bruge interviewet til at både underbygge de ting som Sønderbro siger, samt give noget information inden vi får lavet et andet interview.

	Alle afsnit i rapporten er ellers nye, og ikke læst af Søren.
\end{document}