Denne rapport er skrevet af gruppe DAT1 B2-2 på Datalogistudiet på Aalborg Universitet i forbindelse med studiets P1-projekt. Forløbets titel er "Fra eksisterende software til modeller". Projektet skal afspejle de færdigheder og kompentencer, der stilles som krav, herunder at kunne forstå og gøre rede for anvendte teorier og metoder til analyse og anvendelse af datalogiske begreber inden for programmering samt modellering. Gruppen har i rapporten taget udgangspunkt i Kærbyskolen i Aalborg, som der gennem hele forløbet har været et tæt samarbejde med.

Projektgruppen vil i denne anledning gerne rette en stor tak til følgende institutioner og personer for medvirken i projektet i form af interviews mm.:

Kærbyskolen, ledelsesrepræsentant Jesper Foldberg Nielsen.

