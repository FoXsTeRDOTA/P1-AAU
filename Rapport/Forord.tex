Denne rapport er skrevet af gruppe DAT1B2-2 på Datalogistudiet på Aalborg Universitet i forbindelse med studiets P1-projekt. Forløbets titel er "Fra eksisterende software til modeller". Projektet skal afspejle de færdigheder og kompentencer der stilles som krav, herunder, at kunne forstå og gøre rede for anvendte teorier og metoder til analyse, begreber inden for anvendt programmering samt modellering og projektets kontekstuelle forhold. Gruppen har benyttet sig af Kærbyskolen som case og har haft et tæt samarbejde med skolen.

Projektgruppen vil gerne rette en tak til følgende institutioner og personer for medvirken i projektet med interviews mv.:

Kærbyskolen, ledelsesrepræsentant Jesper Foldberg Nielsen.