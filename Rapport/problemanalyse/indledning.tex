\subsection{Motivation til at arbejde med emnet}
Folkeskolerne er under konstant udvikling, både i form af nedskæringer i budgettet, men senest i form af skolereformen som medførte mange ændringer til indholdet i folkeskolerne. Da folkeskolen er et sådan dynamisk væsen, betyder det også at der er findes mennesker som skal løse opgaven med at få skolernes skemaer til at følge den nyeste lovgivning, uden at have skemaer der overlapper. Da ord som ``heldagsskole'' blev nævnt i forbindelse med skolereformen, blev dette puslespil gjort langt mere komplekst. Skolereformen betød nemlig længere dage på skolen, og derfor flere moduler der skulle placeres i skoleskemaet, for alle klasser.

I forbindelse med skolereformen, var der også usikkerhed omkring om lektiehjælp skulle være en fast del af skoleskemaet. Hvis dette var tilfældet skulle der afsættes tid til både lærere og og elever til dette, men hvis det ikke var påkrævet af loven, skulle timerne lægges anderledes. Dette betød i praksis at skoler blev nødt til at lægge to skemaer, et med skemalagt lektiecaf\'e og et uden. 

I gruppen begyndte vi at snakke hvilke parametre der spiller ind i planlægningen af et skema, og vi undrede os over hvordan man kunne effektivisere en skemalæggers opgave.

Den undren i gruppen, over hvordan man griber en sådan opgave an, har resulteret i vores initierende problem, som lyder: ``Hvilke udfordringer opstår der i forbindelse med skemalægningen i folkeskolen?''.

