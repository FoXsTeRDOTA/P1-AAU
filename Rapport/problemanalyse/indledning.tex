\subsection{Motivation til at arbejde med emnet}
Folkeskolerne er under konstant udvikling, både i form af nedskæringer i budgettet, men senest i form af skolereformen som medførte mange ændringer til indholdet i folkeskolerne. Da folkeskolen er et dynamisk væsen, betyder det også at der findes mennesker som skal løse opgaven med at få skolernes skemaer til at følge den nyeste lovgivning, uden at have skemaer der overlapper. Da ord som ``heldagsskole'' blev nævnt i forbindelse med skolereformen, blev dette puslespil gjort langt mere komplekst. Skolereformen betød nemlig længere dage på skolen, og dermed flere moduler der skulle placeres i skoleskemaet.

I forbindelse med skolereformen, var der også usikkerhed omkring om lektiecaf\'e skulle være en fast del af skoleskemaet. Hvis dette var tilfældet, skulle der afsættes tid for både lærere og elever til dette, men hvis det ikke var påkrævet af loven, skulle timerne lægges anderledes. Dette betød i praksis at folkeskolerne blev nødt til at lægge to skemaer, et med skemalagt lektiecaf\'e og et uden [KILDE]. Det blev dog besluttet i sommerferien 2015, at lektiecafeen skulle være obligatorisk\cite{Lektiecafe} 

I gruppen begyndte vi at snakke om hvilke parametre der spiller ind i planlægningen af et skema, og vi undrede os over hvordan man kunne effektivisere en skemalæggers opgave. Ikke kun så det ville blive nemmere for skemalægningen, men også så skemaet ville kunne tilpasses efter forskellige interesser fra skolerne af.

Den undren i gruppen, over hvordan man griber en sådan opgave an, har resulteret i vores initierende problem, som lyder: ``Hvilke udfordringer opstår der i forbindelse med skemalægningen i folkeskolen?''.

