\subsection{Brugerovervejelser}
\label{brugerovervejelser}
Da dette projekt handler om at udvikle et system til at hjælpe skolerne med deres skemalægning, nytter det ikke noget at lave et system, de ikke kan gennemskue og dermed bruge. Derfor er det relevant at kigge på brugernes færdigheder i forhold til at benytte informationssystemer.

På figur \ref{fig:docendo_skema} er skemalægningsprogrammet Docendo vist. Dette program blev tidligere brugt på Kærbyskolen, og som det ses af figuren, bruges drag\&drop-princippet til at indsætte moduler i skemaet. Dette gør programmet meget intuitivt og let at bruge - noget som Kærbyskolen efterspørger. I interviewet med Kærbyskolen fortalte skemalæggeren, at programmet skulle være meget let at bruge for, at det var en god løsning for skolen.

\begin{figure}[h!]
	\centering
	\includegraphics[width=\textwidth]{Docendo}
	\captionsource{Docendos skemalægningsprogram}{\url{https://docendo.dk/folkeskole.html}}
	\label{fig:docendo_skema}
\end{figure}

På skoler er der flere forskellige systemer der skal arbejde sammen for, at alt fungerer. Skemaet skal eksempelvis lægges ind på nettet på SkoleIntra, så eleverne kan få adgang til det. Derfor er det vigtigt, at et eventuelt program, der laves til at hjælpe med denne skemalægning, er kompatibelt med disse allerede eksisterende programmer, så der ikke skal manuelt arbejde til at konvertere filer fra det ene system til det andet. Heldigvis bruges der i de fleste programmer CSV-filer, hvilket er tekstfiler med kommaseparerede værdier, som så kan læses af de respektive programmer. Det der gør denne filtype nem at arbejde med, er, at vi ved at udskrive informationer omkring skemaet i forskellige rækkefølger, men i samme filformat, kan eksportere til alle disse programmer.

Ved at lave en måde at eksportere på til de forskellige formater, kan skemaet tages direkte fra vores program, og sættes ind i de andre programmer. Dette betyder for skolen, at programmet potentielt kun skal bruge informationer omkring skolen og lærere, og så kan den selv lave et skema, der er klar til, at de kan bruge det direkte.

Interviewet med Tingstrup skole fortalte, at det ikke var nødvendigt med en decideret grafisk brugerflade, så længe at der var noget som fortalte brugeren, hvad programmet lavede. Det vil sige, at det ikke er nødvendigt at have flot grafik, så længe brugeren kan se programmets status. 

Dermed er kravet fra de to skoler et system, som ikke behøves at have en grafisk brugerflade, men skal være nemt at bruge, samtidig med at det hele tiden skal oplyse om, hvor langt det er, så brugeren (skemalæggeren) kan se, om systemet er gået i stå, eller om der er opstået problemer. Når programmet skal laves, er det altså vigtigt, det holdes for øje, at brugerne af systemet ikke nødvendigvis ved hvad de skal, og at programmet derfor skal guide dem igennem processen med at indtaste data.