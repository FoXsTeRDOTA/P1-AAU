\subsection{Kærbyskolen}
\label{Kaerbyskolen}
Vi har interviewet en folkeskole i Aalborg kommune, Kærbyskole, for at kunne danne et skema, som vil opfylde behovene hos den ene skole. Men for at finde ud af hvad andre skoler bruger af systemer og hvad de ville finde fornuftigt at systemet indeholder og hvad kunne bruges for at optimere deres nuværende system, fik vi også interviewet Tingstrup skole i Thisted.

Vores fokus ligger på Kærbyskolen i Aalborg, som vi har valgt som vores case.

Skolen består af omkring 75 medarbejder, som er uddelt blandt ledelse, børnehaveklasseledere, lærere, pædagoger, teknisk administrativt personale og rengøringspersonale. Ud af de 75 medarbejder er 33 af dem lærere.

Kærbyskolen har elever fra børnehaveklassen til 6. klasse og på hver årgang findes to klasser. Udover det har Kærbyskolen fire specialklasser for børn og unge inden for autisme, som består af 32 elever. Skemaer til resten af skolen, udarbejdes normalt en gang pr. år. Tilsammen har skolen omkring 335 elever.

En del af skolens undervisning foregår udenfor skolens områder, da de mangler lokaler til speciale fag, som kræver lidt mere end et almindeligt klasseværelse, som idræt-, hjemkundskab- og sløjtfaciliter. Skolen har en aftale med den lokale kulturskole, hvor de har mulighed for at låne lokaler efter aftale med kulturskolen.

Valget af netop Kærbyskolen som vores case er taget på baggrund af skolens overkommelige størrelse, dens tætte placering i forhold til uddannelsesstedet og skolens store interesse for projektet samt deres vilje til at assistere og videregive kildemateriale.

\subsubsection{Skemaer}
\label{Skemaer}
Børnene i specialklasserne får en individuel struktureret undervisning og får udarbejdet en undervisningsplan tilrettet til hvert barns behov\cite{j_klasser}.
Skemaer for 0. til 6. klasses elever har tidligere været udarbejdet en gang om året, men det var for 2015 planlagt, at skemaet skulle ændres fire gange i løbet af året. Dette blev dog aldrig en realitet og det er nu ubevidst, om skemaet bliver ændret én gang eller slet ikke.

\subsubsection{Reformens indflydelse}
\label{Reformens_indflydelse}
Kærbyskolen sigter efter at følge de mål skolereformen har introduceret pr. 1. august 2014. Ledelsen af Kærbyskole har bestemt, at hverdagen på skolen skal være en varieret og spændende skoledag.

Skolereformen betyder, at de nye skemaer skal tage højde for en længere skoledag. I denne nye længde af dage, skal den nye skoledag have tid til flere timer i grundlæggende fag, som dansk og matematik samt tid til mere motion. Fremmedsprog skal også sættes ind i skemaet allerede fra 1. klasse. Der er også kommet obligatoriske lektiecaféer, hvor eleverne kan få hjælp fra lærerne. Lektiecaféen bliver implementeret, som et normalt fag og vil normalt komme til at ligge sidst på dagen på grund af aftalen med kulturskolen\cite{kaerby_skolereform}.

\subsubsection {Nuværende skemalægning system}
Tidligere har skolen implementeret et system, hvor lærene skrev hver især ønsker og gav dem til en person som skulle danne skemaer baseret på baggrund af alle ønskerne. Men dette var ikke optimalt, da det ført til at lærerne var utilfreds, at deres forberedelsestid blev spredt ud over hele dagen, i stedet for én gang, hvor de kan nå det som gøre at de er bedst forberedt til timerne.. Lærerne ønskede at have mindst 60 minutters forberedelse af gangen, da det giver dem den bedste mulighed for at være klar til at undervise børnene.

Dette system fungerende ikke godt for lærerne, så Kærbyskolen bestemt sig for at ændre det. Sidste gang de skulle ligge skema, var alle medarbejdere, som skulle have et skema, involveret i skemalægningsprocessen. Som det første, begyndt de at diskutere hvilke områder de ville priotere og ud fra det, ville de så arbejde ud fra de områder. Mange medarbejdere var enige om at pædagogiske principper skulle være det man tog udgangspunkt i, men havde svært med at opfylde det, grundet diverse begrænsninger, som lokalemangel og bindinger til eksterne interesser. På forhånd var der allerede bestemt få ting, som at 1. klasse skulle have billedkunst om mandagen, men alle andre detaljer og de mere præcise tidspunkter skulle der være enighed om. 

Dette tog meget lang tid, da de var ca. 60 medarbejdere om at lave skemaet. Tanken var at det skulle tage 3 til 4 timer at nå frem til det skema, men det tog omkring 8 timer for at alle ansatte kunne blive enige om et skema. Oprindeligt, skulle der, i løbet af et år, være 4 skemaer. Men da det tog 8 timer at lægge skemaet, var der ingen der har taget initiativ til at der skulle laves et nyt. 


