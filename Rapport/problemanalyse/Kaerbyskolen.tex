\subsection{Kærbyskolen}
\label{Kaerbyskolen}
Vi har interviewet en folkeskole i Aalborg kommune, Kærbyskole, for at kunne danne et skema, som vil opfylde behovene hos den ene skole. Men for at finde ud af hvad andre skoler bruger af systemer og hvad de ville finde fornuftigt at systemet indeholder og hvad kunne bruges for at optimere deres nuværende system, fik vi også interviewet Tingstrup skole i Thisted.

Vores fokus ligger på Kærbyskolen i Aalborg, som vi har valgt som vores case.

Skolen består af omkring 75 medarbejder, som er uddelt blandt ledelse, børnehaveklasseledere, lærere, pædagoger, teknisk administrativt personale og rengøringspersonale. Ud af de 75 medarbejder er 33 af dem lære.

Kærbyskolen går fra børnehaveklassen til 6. klasse. Og på hver klassetrin har de to klasser. Udover det har Kærbyskolen fire specialklasser for børn og unge inden for autisme, som består af 32 elever. Skemaer til resten af skolen, udarbejdes normalt en gang pr. år. Tilsammen har skolen omkring 335 elever.

En del af skolens undervisning foregår udenfor skolens områder, da de mangler lokaler til speciale fag, som kræver lidt mere end et almindeligt klasseværelse, som idræt-, hjemkundskab- og sløjtfaciliter. Skolen har også en aftale med den lokale kulturskole, hvor de har mulighed for at låne lokaler efter aftale med kulturskolen.

Valget af netop Kærbyskolen som vores case er taget på baggrund af skolens overkommelige størrelse, dens tætte placering i forhold til uddannelsesstedet og skolens store interesse for projektet samt deres vilje til at assistere og videregive kildemateriale.

\subsubsection{Skemaer}
\label{Skemaer}
Børnene i specialklasserne får en individuel struktureret undervisning og får udarbejdet en undervisningsplan tilrettet til hvert barns behov\cite{j_klasser}.
Skemaer for 0. til 6. klasses elever har tidligere været udarbejdet en gang om året, men det var for 2015 planlagt, at skemaet skulle ændres fire gange i løbet af året. Dette blev dog aldrig en realitet og det er nu ubevidst, om skemaet bliver ændret én gang eller slet ikke.

\subsubsection{Reformens indflydelse}
\label{Reformens_indflydelse}
Kærbyskolen sigter efter at følge de mål skolereformen har introduceret pr. 1. august 2014. Ledelsen af Kærbyskole har bestemt, at hverdagen på skolen skal være en varieret og spændende skoledag.

Skolereformen betyder, at de nye skemaer skal tage højde for en længere skoledag. I denne nye længde af dage, skal den nye skoledag have tid til flere timer i grundlæggende fag, som dansk og matematik samt tid til mere motion. Fremmedsprog skal også sættes ind i skemaet allerede fra 1. klasse. Der er også kommet obligatoriske lektiecaféer, hvor eleverne kan få hjælp fra lærerne. Lektiecaféen bliver implementeret, som et normalt fag og vil normalt komme til at ligge sidst på dagen på grund af aftalen med kulturskolen\cite{kaerby_skolereform}.

