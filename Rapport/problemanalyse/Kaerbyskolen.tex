\subsection{Kærbyskolen}
\label{Kaerbyskolen}
Vi har interviewet en folkeskole i Aalborg kommune, Kærbyskolen, for at undersøge hvilke udfordringer en skole har under udarbejdelsen af skoler. For at finde ud af om Kærbyskolens udfordringer er generelle for folkeskoler, har vi også interviewet Tingstrup skole. Med disse to interviews kan vi så undersøge situationen omkring skemalægningen i folkeskolen, og se hvordan situationen bliver tacklet i dag.


Vores fokus i problemanalysen og problemløsningen ligger på Kærbyskolen i Aalborg, som vi har arbejdet tæt sammen med gennem projektet.

Skolen består af omkring 75 medarbejdere, som er fordelt mellem ledelse, børnehaveklasseledere, lærere, pædagoger, teknisk administrativt personale og rengøringspersonale. Ud af de 75 medarbejdere er 33 af dem lærere.

Kærbyskolen har elever fra børnehaveklassen til 6. klasse og på hver årgang findes to klasser. Udover det har Kærbyskolen fire specialklasser for børn og unge med en diagnose inden for den psykiske lidelse autisme, som består af 32 elever. Skemaer til de ordinære klasser udarbejdes normalt en gang årligt. I alt har skolen omkring 335 elever.

En del af skolens undervisning foregår udenfor skolens områder, da de mangler lokaler til specielle fag, som kræver lidt mere end et almindeligt klasseværelse, disse lokaler er blandt andet idræt-, hjemkundskab- og sløjdlokaler. Skolen har en aftale med den institutionen Kulturskolen, som de har mulighed for at låne lokaler af.

Valget af netop Kærbyskolen som vores case er taget på baggrund af skolens overkommelige størrelse, dens tætte placering i forhold til uddannelsesstedet og skolens store interesse for projektet samt deres vilje til at assistere og videregive kildemateriale.

\subsubsection{Skemaer}
\label{Skemaer}
Børnene i specialklasserne får en individuel struktureret undervisning og får udarbejdet en undervisningsplan tilrettet til hvert barns behov\cite{j_klasser}.
Skemaer for 0. til 6. klasses elever har tidligere været udarbejdet en gang om året, men det var for 2015 planlagt, at skemaet skulle ændres fire gange i løbet af året. Dette blev dog aldrig en realitet og det er nu uvist, om skemaet bliver ændret \'en gang i løbet af året, eller slet ikke.

\subsubsection{Reformens indflydelse}
\label{Reformens_indflydelse}
Kærbyskolen sigter efter at følge de mål skolereformen har introduceret pr. 1. august 2014. Ledelsen af Kærbyskole har bestemt, at hverdagen på skolen skal være en varieret og spændende skoledag.

Skolereformen betyder, at de nye skemaer skal tage højde for en længere skoledag. I disse længere dage, skal den nye skoledag have tid til flere timer i grundlæggende fag som dansk og matematik, samt mere tid til motion. Fremmedsprog skal også sættes ind i skemaet allerede fra 1. klasse. Der er også kommet obligatoriske lektiecaf\'eer, hvor eleverne kan få hjælp fra lærerne til at løse opgaver i de forskellige fag. Lektiecaf\'een bliver implementeret, som et normalt fag og vil normalt komme til at ligge sidst på dagen på grund af aftalen med kulturskolen\cite{kaerby_skolereform}.

\subsubsection {Nuværende skemalægning system}
Tidligere har skolen brugt et system, hvor lærerne hver især skrev ønsker til skemaet, og gav dem til en person som skulle danne skemaer baseret på baggrund af alle ønskerne. Men dette var ikke optimalt, da det førte til at lærerne var utilfredse, at deres forberedelsestid blev spredt ud over hele dagen, i stedet for én gang, hvor de kan nå det som gøre at de er bedst forberedt til timerne\cite{interview_Kaerby}. Lærerne ønskede at have mindst 60 minutters forberedelse af gangen, da det giver dem den bedste mulighed for at være klar til at undervise børnene.

Dette system fungerende ikke godt for lærerne, så Kærbyskolen bestemte sig for at ændre det. Sidste gang de skulle ligge skema, var alle medarbejdere, som skulle have et skema, involveret i skemalægningsprocessen. Som det første diskuterede de hvilke parametre de ville prioritere i skemaet. Ud fra disse fokusområder, ville de så lave et skema der opfylder flest muligt af deres krav. Mange medarbejdere var enige om at pædagogiske principper skulle være det man tog udgangspunkt i, men i praksis var det meget at tage højde for grundet begrænsninger som lokalemangel og bindinger til eksterne samarbejdspartnere. Før skemalægningen var der allerede bestemt få ting, som at 1. klasse skulle have billedkunst om mandagen, grundet begrænsninger til Kulturskolens lokaler, men alle andre detaljer og de mere præcise tidspunkter skulle der være enighed om. 

Dette tog meget lang tid, da de var ca. 60 medarbejdere om at lave skemaet. Tanken var at det skulle tage 3 til 4 timer at nå frem til det skema, men det tog omkring 8 timer for at alle ansatte kunne blive enige om et skema. Oprindeligt, skulle der, i løbet af et år, være 4 skemaer. Men da det tog 8 timer at lægge skemaet, var der ingen der har taget initiativ til at der skulle laves et nyt, selvom dette efter planen skulle være gjort til efterårsferien i uge 42. 


