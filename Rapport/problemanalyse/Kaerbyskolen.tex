\subsection{Kærbyskolen}
\label{Kaerbyskolen}
Vi har interviewet en folkeskole i Aalborg kommune, Kærbyskolen, for at undersøge hvilke udfordringer en skole har under udarbejdelsen af skemaer. For at finde ud af om Kærbyskolens udfordringer er generelle for folkeskoler, har vi også interviewet Tingstrup skole I Thisted. Med disse to interviews er det muligt for os at undersøge situationen omkring skemalægningen i folkeskolen og se hvordan, situationen bliver taklet i dag.


Vores fokus i problemanalysen og problemløsningen ligger på Kærbyskolen i Aalborg, som vi igennem projektet har haft et tæt samarbejde med.

Skolen består af omkring 75 ansatte, som er fordelt mellem ledelse, børnehaveklasseledere, lærere, pædagoger, teknisk administrativt personale og rengøringspersonale. Ud af de ansatte er 33 af dem lærere.

Kærbyskolen har elever fra børnehaveklasse til 6. klasse og på hver årgang findes to klasser. Herudover har Kærbyskolen fire specialklasser, som tilsammen består af 32 elever\cite{KaerbyskolensElevtal}, for børn og unge med en diagnose inden for den psykiske lidelse autisme. Skemaer til de ordinære klasser udarbejdes normalt en gang årligt. I alt har skolen 335 elever.

En del af skolens undervisning foregår udenfor skolens eget område, da de mangler lokaler til specielle fag, som kræver specifikke faciliteter. Disse lokaler er blandt andet idræts-, hjemkundskabs- og sløjdlokaler. Skolen har en aftale med institutionen Kulturskolen, som de har mulighed for at låne lokaler af.

Valget af netop Kærbyskolen som vores case er truffet på baggrund af skolens overkommelige størrelse, dens tætte placering i forhold til vor uddannelsessted og skolens store interesse for projektet samt deres vilje til at assistere og videregive materiale.

\subsubsection{Skemaer}
\label{Skemaer}
Børnene i specialklasserne får en individuel struktureret undervisning og får udarbejdet en undervisningsplan tilrettet til hvert barns behov\cite{j_klasser}.
Skemaer for 0. til 6. klasses elever har tidligere været udarbejdet \'en gang om året, men det var for 2015 planlagt, at skemaer skulle laves fire gange i løbet af året. Dette blev dog aldrig en realitet, og det er nu uvist, om skemaet bliver ændret \'en gang i løbet af året, eller slet ikke.

\subsubsection{Reformens indflydelse}
\label{Reformens_indflydelse}
Kærbyskolen sigter efter at følge de mål, skolereformen har introduceret pr. 1. august 2014. Ledelsen af Kærbyskole har bestemt, at hverdagen på skolen ønskes opfattet som en varieret og spændende skoledag.

Skolereformen betyder, at de nye skemaer skal tage højde for en længere skoledag. I disse længere dage, skal den nye skoledag have plads til flere timer i grundlæggende fag som dansk og matematik samt mere tid til motion. Fremmedsprog skal også sættes ind i skemaet allerede fra 1. klasse. Der er også kommet obligatoriske lektiecaf\'eer, hvor eleverne kan få hjælp fra lærerne til at løse opgaver i de forskellige fag. Lektiecaf\'een bliver indført, som et normalt fag og vil normalt komme til at ligge sidst på dagen på grund af aftalen med Kulturskolen\cite{kaerby_skolereform}.

\subsubsection {Nuværende skemalægningsystem}
Tidligere har skolen brugt et system, hvor lærerne hver især skrev ønsker til skemaet, og gav dem til en skemalægger, som så skulle danne skemaer baseret på baggrund af alle ønskerne. Men dette følte Kærbyskolen ikke var optimalt, da det var ineffektivt, førte til at lærerne var utilfredse med, at deres forberedelsestid blev spredt ud over hele dagen, i stedet for et samlet modul, hvor de kan fokusere på forberedelsen og fordybe sig\cite{interview_Kaerby}. Lærerne ønskede at have mindst 60 minutters forberedelse af gangen, da det giver dem den bedste mulighed for at arbejde koncentreret, fordybe sig og blive klare til at undervise eleverne.

Dette system fungerende ikke godt for lærerne, så Kærbyskolen bestemte sig for at ændre det. Sidste gang de skulle ligge skemaer, var alle medarbejdere, som skulle have et skema, derfor involveret i skemalægningsprocessen. Som det første diskuterede de hvilke parametre de ville prioritere i skemaet. Ud fra disse fokusområder, ville de så lave et skema, der opfylder flest mulige krav. Mange medarbejdere var enige om, at pædagogiske principper skulle være det man tog udgangspunkt i, men i praksis var det svært at tage højde for grundet begrænsninger som lokalemangel og bindinger til eksterne samarbejdspartnere. Før skemalægningen var der allerede bestemt nogle ting og altså dannet bindinger, så som at 1. klasse skulle have billedkunst om mandagen, grundet begrænsninger af adgang til Kulturskolens lokaler, men alle andre detaljer og de mere præcise tidspunkter skulle der være enighed om. 

Dette tog meget lang tid, da de var ca. 60 medarbejdere om at lave skemaet. Tanken forinden var, at det skulle tage tre til fire timer at nå frem til dette skema, men det tog omkring otte timer for alle ansatte at blive enige om et skema. Oprindeligt skulle der i løbet af et år dannes fire skemaer. Da det tog otte timer at lægge skemaet, er der dog siden da ingen, som har taget initiativ til at lave et nyt selvom dette efter planen, skulle have være gjort til efterårsferien.


