\subsection{State of the Art}
Indenfor skemalægning findes allerede nogle løsninger på de udfordringer, som nemt opstår indenfor skemalægning. Et af disse er USA Schedulers program kaldet School Master Scheduling Software\cite{USAS}, et program som tager de studerende i betragtning først og fremmest. Programmet er i stand til at lave funktionelle skemaer ud fra de studerenes ønsker, analysere de mest optimale løsninger ud fra ønskerne, og ud fra dette, undgår konflikter så som 2 moduler, der ligger oveni hinanden så studerende, der deltager aktivt i begge, bliver nødt til at vælge den ene fra.
Et andet stykke software kaldet Mimosa Scheduling Software\cite{Mimosa}, hvilket fokuserer mere på de grundlæggende udfordringer så som overbooking og nemt at kunne ændre på skemaerne, hvis der skulle komme en uventet udfordring. Det er i stand til at lave skemaer for lærere, elever, klasser og lokaler, alt det som enhver dansk folkeskole har brug for. Institutionerne henter programmet, laver deres egne ressourcer, hvilket vil sige lærere, elever, lokaler osv. og derefter fordeler de forskellige ressourcer ud over dagene. Programmet vil så fortælle om dette kan lade sig gøre, og hvis dette ikke er tilfældet, vil det fortælle hvorfor\cite{MimosaTutorial}. I modsætning til School Master Scheduling, er Mimosas program mere henvendt til skemalæggerne, og ikke så meget til de studerende. Begge programmer løser dog ikke selve skemalægningsprocessen, da dette stadig gøres manuelt af skemalæggeren, eller de individuelle studerende for deres eget skema, og ser man over top 10 mest anbefalede skemalægningsprogrammer\cite{top10Schedulers}, er de ikke rangeret ud fra automatisk skemalægning. Det, som kommer tættest på, vil være Auto-Scheduler, men denne funktionalitet virker kun for medarbejdere ud fra en historik, og er derfor ikke automatisk skemalægning. Besøger man et af de ti's websteder, får man heller ingen information om de er i stand til netop dette. Et program, som vil være i stand til helt automatisk at lave skemaer til alle på for eksempel en dansk folkeskole, kun ud fra de ansatte, eleverne og klasselokalerne som er til rådighed, vil derfor være innovativt.
