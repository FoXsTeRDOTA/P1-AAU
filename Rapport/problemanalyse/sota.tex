\subsection{State of the Art}
Indenfor skemalægning findes allerede nogle løsninger på de udfordringer, som nemt opstår indenfor skemalægning. Et af disse er USA Schedulers program kaldet School Master Scheduling Software\cite{USAS}, et program som tager de studerende i betragtning først og fremmest. Programmet er i stand til at lave funktionelle skemaer ud fra de studerenes ønsker, analysere de mest optimale løsninger ud fra ønskerne, og ud fra dette, undgår konflikter så som 2 moduler, der ligger oveni hinanden så studerende, der deltager aktivt i begge, bliver nødt til at vælge den ene fra.
Et andet stykke software kaldet Mimosa Scheduling Software\cite{Mimosa}, hvilket fokuserer mere på de grundlæggende udfordringer så som overbooking og nemt at kunne ændre på skemaerne, hvis der skulle komme en uventet udfordring. Det er i stand til at lave skemaer for lærere, elever, klasser og lokaler, alt det som enhver dansk folkeskole har brug for.