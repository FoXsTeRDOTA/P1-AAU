\subsection{SWOT-analyse}
For at forstå \school bedre, kan vi lave en SWOT-analyse af skolen som en virksomhed. En SWOT-analyse er en model, man kan benytte for at analysere et firmas Styrker, Svagheder, Muligheder og Trusler (Strengths, Weaknesses, Opportunities og Threats). På denne måde ønsker vi at opnå en dybere forståelse af skolens virkemåde samt se, hvor vi med et program kan forbedre skolens muligheder.
\begin{table}[!h]
	\centering
	\resizebox{\textwidth}{!}{%
	\begin{tabular}{|l|l|l|}
		\hline
		{\color[HTML]{000000} }                          & {\color[HTML]{000000} Styrker}                                                                                           & {\color[HTML]{000000} Svagheder}                                                                                                                                                                          \\ \cline{2-3} 
		{\color[HTML]{000000} }                          &                                                                                                                          &                                                                                                                                                                                                           \\
		{\color[HTML]{000000} }                          &                                                                                                                          &                                                                                                                                                                                                           \\
		{\color[HTML]{000000} }                          &                                                                                                                          &                                                                                                                                                                                                           \\
		\multirow{-5}{*}{{\color[HTML]{000000} Intern}}  & \multirow{-4}{*}{\begin{tabular}[c]{@{}l@{}}Tæt samarbejde mellem de ansatte\\ Fokus på pædagogisk læring.\end{tabular}} & \multirow{-4}{*}{\begin{tabular}[c]{@{}l@{}}Splittet forberedelsestid pga. indlagte vikartimer\\ i skemaet.\\ Bruger ofte eksterne vikarer pga. stressede lærere\\ Ikke så teknisk vidende.\end{tabular}} \\ \hline
		{\color[HTML]{000000} }                          & {\color[HTML]{000000} Muligheder}                                                                                        & {\color[HTML]{000000} Trusler}                                                                                                                                                                            \\ \cline{2-3} 
		{\color[HTML]{000000} }                          &                                                                                                                          &                                                                                                                                                                                                           \\
		{\color[HTML]{000000} }                          &                                                                                                                          &                                                                                                                                                                                                           \\
		\multirow{-4}{*}{{\color[HTML]{000000} Ekstern}} & \multirow{-3}{*}{\begin{tabular}[c]{@{}l@{}}Programmer som kan lætte skema-\\ lægningsbyrden.\end{tabular}}              & \multirow{-3}{*}{\begin{tabular}[c]{@{}l@{}}Afhængig af Kulturhuset for manglende faciliteter.\\ Konstant ændring af regler, \\ har effekt på skemalægning.\end{tabular}}                                 \\ \hline
	\end{tabular}
	}
\caption{SWOT Analyse}
\label{swot-analyse}
\end{table}

\subsubsection*{Styrker}
I form af Kærbyskolens mindre størrelse på kun 335 elever og 33 lærere\cite{Kaerbyskolens-laerere}, er der muligheder for et tættere arbejde mellem de ansatte. Dette kom eksempelvis til udtryk, under skemalægningen, hvor alle lærere og pædagoger arbejdede sammen om at lægge et skema.

Selvom det ifølge skolen selv ikke var den største succes at lægge skemaet på denne måde, viser det alligevel hvor tæt et samarbejde de ansatte kan have med hinanden på skolen.

For skolen giver dette tætte samarbejde en højere kvalitet, da man ved at snakke med alle parter sikrer sig, at alle behov på skolen bliver hørt. Hvis vi igen tager udgangspunkt i den fælles skemalægning betød samarbejdet, at alle lærere og dermed alle klassetrin blev repræsenteret under skemalægningen, og at skemaerne derfor i mange henseender blev gode for samtlige klasser på skolen. 

En anden styrke ved skolen er emnet som vi har kigget på i afsnit \ref{paedagogisk_laering}, nemlig pædagogisk læring. Skolen vil i deres skemaer tage højde for pædagogiske overvejelser, og dermed udmærke sig i forhold til andre skoler. Grundet skemalægningens form som vi har beskrevet tidligere, var dette ikke muligt at gøre, men skolen har et ønske om at finde nye skemalægningsmetoder, og dermed kunne tage højde for de pædagogiske overvejelser.

\subsubsection*{Svagheder}
Ved skolens nuværende model for skemalægning, er der nogle svagheder. Der er lagt timer af i lærernes skema til, at de kan træde ind som vikar for andre lærere. På grund af skemaerne er deres forberedelsestid meget splittet op, hvilket gør, at lærerene bliver stressede, hvis de skal nå at forberede sig til at være vikar samtidig med, at de også skal undervise deres egne klasser.

Så selvom der er aflagt timer i skemaet til, at de kan være vikarer, vælger skolen som regel at bruge eksterne vikarer af hensyn til lærerene på skolen. Dette er en svaghed, som kunne afhjælpes ved at skemaet blev lagt anderledes.

Vi har i afsnit \ref{brugerovervejelser} kigget nærmere på de brugerovervejelser som gør sig gældende, når der skal udviklet et system til skolen. Her blev det tydeligt at de ansatte på skolen havde meget teknisk snilde, og at programmerne derfor skulle være meget lette at bruge, for at skolen vil få noget ud af dem. Dette skaber selvfølgelig nogle svagheder for skolen, da mere avancerede funktioner i eventuelle programmer, kunne kræve en større indsigt i programmet for at kunne benytte.

\subsubsection*{Muligheder}
En stor mulighed for skolen er den teknologiske udvikling. Der komer løbende nye programmer på markedet der udmærker sig inden for et emne omkring skemalægning. Disse programmer og teknologier har mulighed for at lette noget af den arbejdsbyrde der er i skemalægningen for Kærbyskolen. Måske kan programmerne ikke hver især ligge det perfekte skema, men programmer som Docendo som vi beskrev i afsnit \ref{sota}, kan hjælpe skolen med at i hvert fald kontrollere deres skemaer, for hurtigere at tjekke at skemaet opfylder skolens krav.

\subsubsection*{Trusler}
Kærbyskolen har ikke lokaler nok på skolen til alle dennes fag. Det vil sige, at lokaler til fag som idræt, musik og sløjd bliver lånt andetsteds. Skolen låner lokaler fra institutionen Kulturskolen, og skolen opnår dermed faciliteter til at undervise i disse fag. Denne afhængighed af Kulturskolen skaber dog nogle udfordringer og trusler mod Kærbyskolen. Lokalerne på kulturskolen er ikke altid ledige, og når skemaerne skal lægges, skal der tages højde for dette. 

Måden hvorpå skemalæggeren tog højde for dette var ved at sætte nogle faste intervaller, hvor hver klasse havde mulighed for at bruge lokalerne. Dette satte nogle store begrænsninger i forhold til skemalægningen, hvor man nu mistede megen frihed i forhold til placeringen af timerne på grund af kravene til at låne lokaler. Til det møde hvor alle de ansatte i fællesskab skulle ligge skemaer, kom skemalæggeren med en række ufærdige skemaer, hvorpå disse intervaller var lagt ind, og ud fra dette skulle de så fylde resten af skemaet ud.  

Antallet af lokaler er dog ikke den eneste trussel mod skolen. Vi har i afsnit \ref{Lovgivning_og_regler} og afsnit \ref{Reformen} kigget på de love der spiller ind i forhold til skemalægningen, samt kigget nærmere på reformens indvirkning, og vi kan se at reglerne er i konstant ændring. Det samme gælder derfor også for skemalægningsprocessen. Dette skaber for skolen en trussel, da et system skal være meget fleksibelt, hvis de skal kunne sætte deres lid til at programmet vil forsætte med at fungere. Hvis vi kæder dette sammen med skolens svaghed i form af en manglende teknisk snilden, har skolen altså brug for et fleksibelt program, som er nemt at bruge.
