\subsection{SWOT-analyse}
For at forstå \school bedre, kan vi lave en SWOT-analyse af skolen som en virksomhed. En SWOT-analyse er en model man kan benytte for at analysere et firmas Styrker, Svagheder, Muligheder og Trusler (Strengths, Weaknesses, Opportunities og Threats). På denne måde ønsker vi at forstå mere om skolens virkemåde, samt se hvor vi med et program kan forbedre skolens muligheder.

\subsubsection*{Styrker}
I form af Kærbyskolens mindre størrelse på kun 335 elever og 33 lærere\cite{Kaerbyskolens-laerere}, er der muligheder for et tættere arbejde mellem de ansatte. Dette kom eksempelvis til udtryk, under skemalægningen hvor alle lærere og pædagoger arbejdede sammen om at lægge et skema.

Selvom det ifølge skolen selv ikke var den største succes at lægge skemaet på denne måde, viser det alligevel hvor tæt et samarbejde de ansatte kan have med hinanden på skolen.

For skolen giver dette tætte samarbejde en højere kvalitet, da man ved at snakke med alle parter, sikrer sig at alle behov på skolen bliver hørt. Hvis vi igen tager udgangspunkt i den fælles skemalægning, betød samarbejdet at alle lærere og dermed alle klassetrin blev repræsenteret under skemalægningen, og at skemaerne derfor blev gode for samtlige klasser på skolen. 

\subsubsection*{Svagheder}
Ved skolens nuværende model for skemalægning, er der nogle svagheder. Der er lagt timer af i lærernes skema til at de kan træde ind som vikar for andre lærere. På grund af skemaerne, er deres forberedelsestid meget splittet op, hvilket gør at lærerene bliver stressede hvis de skal nå at forberede sig til at være vikar, samtidig med at de også skal undervise deres egne klasser.

Så selvom der er aflagt timer i skemaet til at de kan være vikarer, vælger skolen som regel at bruge eksterne vikarer, af hensyn til lærerende på skolen. Dette er en svaghed, som kunne afhjælpes ved at skemaet blev lagt anderledes.

\subsubsection*{Muligheder}
[Ikke skrevet endnu]

\subsubsection*{Trusler}
Kærbyskolen har ikke lokaler nok på skolen, til alle dennes fag. Det vil sige at lokaler til fag som idræt, musik og sløjd bliver lånt andetsteds. Skolen låner lokaler fra institutionen Kulturskolen, og skolen opnår dermed faciliteter til at undervise i disse fag. Denne afhængighed af Kulturskolen skaber dog nogle udfordringer og trusler mod Kærbyskolen. Lokalerne på kulturskolen er ikke altid ledige, og når skemaerne skal lægges, skal der tages højde for dette. 

Måden hvorpå skemalæggeren tog højde for dette, var ved at sætte nogle faste intervaller hvor hver klasse havde mulighed for at bruge lokalerne. Dette satte nogle store begrænsninger i forhold til skemalægningen, hvor man nu mistede megen frihed i forhold til placeringen af timerne, på grund af kravene til at låne lokaler. Til det møde hvor alle de ansatte i fællesskab skulle ligge skemaer, kom skemalæggeren med en række ufærdige skemaer hvorpå disse intervaller var lagt ind, og ud fra dette skulle de så fylde resten af skemaet op. 