\subsection{Lovgivning og regler for skemalægning}
\label{Lovgivning_og_regler}
I driften af en folkeskole er planlægning et vigtigt redskab i opnåelsen af en velfungerende og læringsrig hverdag for både lærere, elever, sekretærer og andet personale. Planlægning er afgørende og et godt skema ligeså. I folkeskolen er hverdagen som hovedregel baseret på de ugeskemaer, der som minimum inden starten af hvert skoleår udarbejdes af skolen og sidenhen følges nøje\cite{interview_Kaerby}. Fungerer dette skema ikke, er der risiko for, at hverdagen ej heller fungerer, og at undervisningsniveauet rammes af det.

Dette er en af grundene til, at der hvert år lægges mange timer i skemalægning på hver eneste folkeskole i landet, og ofte er der tale om et større team\cite{interview_Kaerby, interview_Tingstrup}, som må samle sig om opgaven. Det som gør det så svært at lægge et velfungerende skema, er den lange liste af lovgivninger, regler, krav og bindinger som de skemalæggende teams er nødsagede til at tage hensyn til. Der stilles krav til skolen af det danske undervisningsministerium i form af bl.a. de nationale Fælles Mål\cite{fmaal}, som skal være styrende for undervisningen og er mål for, hvad eleverne skal lære i de enkelte fag, men i særdeleshed folkeskolereformen og bekendtgørelsen af lov om folkeskolen\cite{Lovgivning} er styrende for skemalægningsprocessen. Under denne bekendtgørelse findes blandt andet en offentliggørelse af folkeskolernes minimums- og vejledende timetal (se bilag \ref{TimetalsKrav}) for de enkelte fag og årgange, som skal følges. Dette giver skemalæggeren et udgangspunkt, men binder samtidig og gør processen mindre fleksibel. Desuden må der tages hensyn til lærernes arbejdstider, tid til forberedelse, skolens egne praktiske og pædagogiske krav og ønsker og, i nogle tilfælde, udefrakommende bindinger i form af lån af speciallokaler og så videre.
