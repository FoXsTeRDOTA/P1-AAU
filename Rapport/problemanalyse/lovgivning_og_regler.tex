\subsection{Lovgivning og regler for skemalægning}
\label{Lovgivning og regler}
I driften af en folkeskole er planlægning et utroligt vigtigt redskab i opnåelsen af en velfungerende og læringsrig hverdag for såvel lærere som elever, sekretærer og andet personale. Planlægning er altafgørende og et godt skema ligeså. I folkeskolen er hverdagen som hovedregel baseret på de ugeskemaer, der som minimum inden starten af hvert skoleår udarbejdes af skolen og sidenhen følges nøje. Fungerer dette skema ikke, er der en stor risiko for, at hverdagen ej heller fungerer og undervisningsniveauet rammes af det.

Dette er en af grundene til, at der hvert år lægges utroligt mange timer i skemalægning på hver eneste folkeskole i landet og oftest er der tale om et større team, som må samle sig om opgaven. Hvad der gør det så utroligt svært at lægge et velfungerende skema er den nærmeste uendelige liste af lovgivninger, regler, krav og bindinger som de skemalæggende teams er nødsagede til at tage hensyn til.  Der stilles krav til skolen af det danske undervisningsministerium i form af bl.a. de nationale Fælles Mål \cite{fmaal}, som skal være styrende for undervisningen og er mål for, hvad eleverne skal lære i de enkelte fag, men i særdeleshed folkeskolereformen og bekendtgørelsen af lov om folkeskolen \cite{Lovgivning} er styrende for skemalægningsprocessen. Under denne bekendtgørelse findes nemlig blandt andet en offentliggørelse af folkeskolernes minimums- og vejledende timetal (Bilag 2) for de enkelte fag og årgange, som skal følges. Dette giver skemalæggeren et udgangspunkt, men binder samtidig og gør processen mindre fleksibel. Desuden må der tages hensyn til lærernes arbejdstider, tid til forberedelse, skolens egne praktiske og pædagogiske krav og ønsker og i særdeles mange tilfælde, udefrakommende bindinger i form af lån af speciallokaler og så videre.

% Please add the following required packages to your document preamble:
%\usepackage{graphicx}
\iffalse
\begin{table}[]
	\centering
	\caption{My caption}
	\label{TimetalsKrav}
	\resizebox{\textwidth}{!}{%
		\begin{tabular}{lllllllllllll}
			Timetal (minimumstimetal og vejledende timetal) for fagene &                               &     &     &     &     &     &     &     &     &     &     &                         \\
			Klassetrin                                                 &                               & Bh. & 1.  & 2.  & 3.  & 4.  & 5.  & 6.  & 7.  & 8.  & 9.  & Timetal i alt           \\
			\textbf{A. Humanistiske fag}                               &                               &     &     &     &     &     &     &     &     &     &     &                         \\
			Dansk                                                      & \textit{(minimumstimetal)}    &     & 330 & 300 & 270 & 210 & 210 & 210 & 210 & 210 & 210 & 2.160                   \\
			Engelsk                                                    & \textit{(vejledende timetal)} &     & 30  & 30  & 60  & 60  & 90  & 90  & 90  & 90  & 90  & 630                     \\
			Tysk eller fransk                                          & \textit{(vejledende timetal)} &     &     &     &     &     & 30  & 60  & 90  & 90  & 90  & 360                     \\
			Historie                                                   & \textit{(minimumstimetal)}    &     &     &     & 30  & 60  & 60  & 60  & 60  & 60  & 30  & 360                     \\
			Kristendomskundskab                                        & \textit{(vejledende timetal)} &     & 60  & 30  & 30  & 30  & 30  & 60  &     & 30  & 30  & 300                     \\
			Samfundsfag                                                & \textit{(vejledende timetal)} &     &     &     &     &     &     &     &     & 60  & 60  & 120                     \\
			\textbf{B. Naturfag}                                       & \textit{}                     &     &     &     &     &     &     &     &     &     &     &                         \\
			Matematik                                                  & \textit{(minimumstimetal)}    &     & 150 & 150 & 150 & 150 & 150 & 150 & 150 & 150 & 150 & 1.350                   \\
			Natur/teknik                                               & \textit{(vejledende timetal)} &     & 30  & 60  & 60  & 90  & 60  & 60  &     &     &     & 360                     \\
			Geografi                                                   & \textit{(vejledende timetal)} &     &     &     &     &     &     &     & 60  & 30  & 30  & 120                     \\
			Biologi                                                    & \textit{(vejledende timetal)} &     &     &     &     &     &     &     & 60  & 60  & 30  & 150                     \\
			Fysik/kemi                                                 & \textit{(vejledende timetal)} &     &     &     &     &     &     &     & 60  & 60  & 90  & 210                     \\
			&                               &     &     &     &     &     &     &     &     &     &     &                         \\
			\textbf{C. Praktiske/musiske fag}                          &                               &     &     &     &     &     &     &     &     &     &     &                         \\
			Idræt                                                      & \textit{(vejledende timetal)} &     & 60  & 60  & 60  & 90  & 90  & 90  & 60  & 60  & 60  & 630                     \\
			Musik                                                      & \textit{(vejledende timetal)} &     & 60  & 60  & 60  & 60  & 60  & 30  &     &     &     & 330                     \\
			Billedkunst                                                & \textit{(vejledende timetal)} &     & 30  & 60  & 60  & 60  & 30  &     &     &     &     & 240                     \\
			Håndværk og design samt madkundskab                        & \textit{(vejledende timetal)} &     &     &     &     & 90  & 120 & 120 & 60  &     &     & 390                     \\
			\textbf{D. Valgfag}                                        & \textit{(vejledende timetal)} &     &     &     &     &     &     &     & 60  & 60  & 60  & 180                     \\
			\textbf{E. Årligt minimumstimetal pr. klassetrin}          &                               & 600 & 750 & 750 & 780 & 900 & 930 & 930 & 960 & 960 & 930 & 7.890 ekskl. bh. /8.490
		\end{tabular}
	}
\end{table}
Note: Timetallene er angivet i klokketimer og uden pauser.Note: Bh.: Børnehaveklasse.
\fi