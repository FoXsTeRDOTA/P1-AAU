\subsection{Reformen}
\label{Reformen}
Folkeskolerne i Danmark, har igennem de sidste par år gennemgået en stor forandring. Den danske regering har valgt at ændre systemet for at sørge for, at eleverne får størst mulig udbytte af deres undervisning uanset hvilken baggrund de har, eller hvor fagligt stærke de er.
Dem der står bag denne reform opstiller få klare mål\cite{reformenMaal}:
	\begin{itemize}
		\item Folkeskolen skal udfordre alle elever, så de bliver så dygtige, de kan.
		\item Folkeskolen skal mindske betydningen af social baggrund for de faglige resultater.
		\item Tilliden til og trivslen i folkeskolen skal styrkes gennem blandt andet respekt for professionel viden og praksis.
	\end{itemize}

Reformen trådte i kraft fra skolestarten i 2014. Siden denne skolestart har eleverne fra folkeskolerne haft en længere skoledag end de plejede inden reformen. Den ekstra tid i skolen som eleverne får, betyder, at der er mere tid til, at den enkelte elev kan lære mere. Hvor meget ekstra tid hver elev får afhænger af elevens alder. For eksempel slutter de mindste elever typisk kl. 14 og de ældste omkring kl 15.

Man får også flere timer til to såkaldte ”kernefag”, matematik og dansk, der ses som værende grundlæggende for at kunne forstå og effektivt lære indenfor andre fagområder. Udover flere timer til de mest grundlæggende, meget faglige fag skal motion også implementeres i elevernes skoledag med i gennemsnit 45 minutters daglig bevægelse\cite{reformenBorger}. Udover den ekstra tid til mere undervisning, kommer der også en obligatorisk lektiecafé, hvilket vil indgå i elevernes skemaer. Denne lektiecafé har til måls at sikre, at eleverne får lavet deres lektier og får størst muligt udbytte af at lave disse.

Der kommer også en stor investering i optimering af undervisningen. Undervisningen skal være af en højere kvalitet end forinden reformen og kompetencen af lærere og pædagoger skal øges, så de kan varetage de krav og mål som folkeskolereformen sætter. Undervisningsministeriet siger, at alle lærere skal have undervisningskompetence i fagene de underviser i, og der er afsat ca. 1 milliard kroner fra 2014-2020 til, at lærerne kan få en styrket efteruddannelse. Det vil resultere i, at eleverne kan få mere tid sammen med bedre kvalificerede lærere.

Med indføringen af reformen bliver der også indført nogle regelforenklinger af folksekolelove, så kommunerne får mere frihed til at lave undervisning efter det lokale miljø. Det vil sige, at der kommer flere muligheder for fleksible regler for en skolebestyrelse.
