\subsection{Generalt ved skemalægning}
Eftersom det er lovpligtigt for de offentlige folkeskoler at følge skolereformen, da denne går ind under den danske lovgivning, skal alle landets skoler designe et skema for alle skolens årgange og klasser som stemmer overens med minimumstimetal- og vejledende timetalskravende ved hhv. \ref{TimetalsKrav}. Eftersom der er 200 skoledage om året i en folkeskole\cite{elevers_timetal}, ved hver skole hvor mange timer de ca. ugeligt, skal sætte af til Dansk, Matematik, osv. Alle skoler ligger deres skemaer ud fra restriktioner så som dette, og en af de skoler som gør netop dette, er \school ``\ref{InterviewKaerby}'', som har mange restriktioner pga. det er en lille skole og derfor låner mange af deres specialelokaler så som musiklokaler og idrætfacilliter fra Kulturhuset, som ligger ved siden af skolen. Skolen her havde allerede lagt disse forskellige lektioner ind i skemaerne de skulle til at lægge før de overhovedet gik i gang med at fylde skemaet ud, ud fra minimum og vejledende timetalskravene.

En anden skole vi har interviewet ``\ref{InterviewTingstrup}'', havde ikke de samme restriktioner indenfor lokaler, så de begyndte at ligge skemaerne ud fra timetalskravende. Ligesom ved \school, lagde Tingstrup også skemaerne sammen i grupper, dog en væsentlig mindre gruppe (tre frem for treds). Vi kan ud fra disse to cases, godt regne med at skemalægningen oftest ikke er noget som kun en person står med.

Tidligere havde \school brugt programmet Docendo og Tingstrup havde tidligere brugt Matrix, til at lægge deres skemaer i. Begge skoler ligger deres skemaer ud på Skoleintra så både lærere, elever og forældre, kan se hvilke timer de har og hvornår. Det er en væsentlig ting at både lærer og elever let har adgang til deres dagsorden og for elevernes vedkommende, også deres lektier og afleveringsopgaver, så uanset om skolerne laver deres skema manuelt, eller lægger dem i et stykke software, er det en generalt ting at ligge skemaerne op på en platform som Skoleintra.
