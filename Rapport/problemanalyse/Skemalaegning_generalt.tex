\subsection{Generelt ved skemalægning}
Eftersom det er lovpligtigt for de offentlige folkeskoler at følge skolereformen, da denne går ind under den danske lovgivning, skal alle landets skoler designe et skema for alle skolens årgange og klasser som stemmer overens med minimumstimetal- og vejledende timetalskravende som er vist i tabel \ref{TimetalsKrav}. Eftersom der er 200 skoledage om året i en folkeskole\cite{elevers_timetal}, ved hver skole hvor mange timer de ugenligt, skal sætte af til dansk, matematik, osv. 


%Alle skoler lægger deres skemaer ud fra restriktioner så som dette, og en af de skoler som gør netop dette, er \school (bilag \ref{InterviewKaerby}), som har mange restriktioner pga. det er en lille skole og derfor låner mange af deres specialelokaler så som musiklokaler og idrætfacilliter fra Kulturhuset. Skolen her havde allerede lagt disse forskellige lektioner ind i skemaerne de skulle til at lægge før de overhovedet gik i gang med at fylde skemaet ud, ud fra minimum og vejledende timetalskravene. Disse special timer var vigtige at placere da de havde bestemte restriktioner, der gjorde at hvis man skulle have et udbytte af timen ville det være nædvendigt at ligge dem på tidspunkter hvor de vil have lokaler til det. Så havde man på \school valgt at give opgaven til lærerne selv. Så de delt op i teams, og lavede derfra et skema til de forskellige årgange som de var undervisere for (bilag \ref{InterviewKaerby}). Her var alle skolens ansatte med inde under selve skemalægnings processen. 

%En anden skole vi har interviewet (bilag \ref{InterviewTingstrup}), havde ikke de samme restriktioner indenfor lokaler, så de begyndte at ligge skemaerne ud fra timetalskravende. Ligesom ved \school, lagde Tingstrup også skemaerne sammen i grupper, dog en væsentlig mindre gruppe (tre frem for treds). En anden meget væsentlig forskel mellem disse 2 skoler er, at de på Tingstrup stod 3 ledere for skemalægningen, hvor de på \school havde alle ansatte med til skemalægningen. Vi kan ud fra disse to cases, godt regne med at skemalægningen oftest ikke er noget som kun en person står med.

%Tidligere havde \school brugt programmet Docendo og Tingstrup havde tidligere brugt Matrix, til at lægge deres skemaer i. Begge skoler ligger deres skemaer ud på Skoleintra så både lærere, elever og forældre, kan se hvilke timer de har og hvornår. Det er en væsentlig ting at både lærer og elever let har adgang til deres dagsorden og for elevernes vedkommende, også deres lektier og afleveringsopgaver, så uanset om skolerne laver deres skema manuelt, eller lægger dem i et stykke software, er det en generalt ting at ligge skemaerne op på en platform som Skoleintra.
