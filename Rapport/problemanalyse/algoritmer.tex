\subsubsection{Brute Force}
Vi vil nu undersøge nogle af de teknologier, som et program kunne benytte sig af for at finde frem til et skema. Den mest systematiske løsningen, og måske den løsning der er nemmest af gå til, er brute force-metoden. Brute force går som navnet antyder ud på at bruge rå kraft for at finde løsningen. Måden dette gøres på i en computer, er ved at prøve samtlige muligheder, indtil man når det ønskede resultat\cite{BruteForce}.

I forhold til skemalægning, ville man altså forsøge samtlige sammensætninger af moduler i løbet af ugen, for alle klasse. For at undersøge om dette ville virke, kan vi regne ud hvor lang tid det vil tage at regne samtlige skemaer ud.

$$ (\text{UgentligeModuler})! \times \text{AntalKlasser} $$
\subsubsection{Genetiske algoritmer}
\subsubsection{Monte Carlo / Las Vegas}
\subsubsection{Heuristisk metode}
