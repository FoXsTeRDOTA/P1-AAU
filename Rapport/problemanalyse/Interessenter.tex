\subsection{Interessenter}
Når vi snakker om udfordringerne i forhold til skemalægningen i folkeskolen, er det ganske relevant at kigge på de grupper af mennesker, som har en interesse inden for dette emne. Vi vil derfor i dette afsnit forsøge at klarlægge, hvilke parter kunne have en form for interesse i en eventuel løsning på disse udfordringer.

\subsubsection{Skoleledelsen}
I dag skal der afsættes mange timer til, at en eller flere ansatte på en skole kan lægge skemaer for samtlige klasser. Ved at automatisere denne opgave vil skolen kunne spare nogle af de udgifter, der opstår i forbindelse med skemalægning. Udover at pengene kan spares de antal gange om året, hvor skemaerne lægges, kan programmet bruges igen og igen, hvis der opstår ændringer i strukturen på skolen, og det gamle skema viser sig at være uhensigtsmæssigt. Hvis der for eksempel på en skole sker en lærerudskiftning, og der kommer nye lærere til skolen, mens andre forlader den, er det ikke sikkert, at disse lærere vil kunne undervise i de samme sammensætninger af fag og klasser som de gamle lærere.

I tilfældet med Tingstrup skole, var der til tre ledere afsat otte arbejdsdage til at få skemaet til at gå op. Lederstillinger i folkeskolen har en årsløn på mellem 500.540kr. og 729.549kr.\cite{TR_HAANDBOGEN, Statens_adm}, så otte arbejdsdage for sådanne tre stillinger bliver en enorm samlet udgift for staten taget i betragtning af, at vi har 1.313 folkeskoler i Danmark\cite{UVM-Folkeskoler}. Noget af tiden på disse otte arbejdsdage blev også brugt på seminarer, hvilket man kunne forestille sig stadig ville være nødvendigt, hvis ikke der skulle lægges skemaer. Den overskydende tid som disse ledere skulle tilbringe på skolen med at lave skemaer, kunne nu bruges på andre gøremål i stedet eller helt undværes.

Ved at automatisere skemalægningsopgaven, vil der forhåbentligt kunne opnås en større fleksibilitet i skemaet, da et computersystem har muligheden for at teste mange flere skemaer end \'en person eller \'et team, og dermed også undgå mange af de fejl, som måtte opstå under manuel skemalægning. Ved at skemaerne kan genereres hurtigere ved hjælp af en computer, er det også muligt at indarbejde flere ønsker og begrænsninger i skemaet, uden at det bliver uoverskueligt for skemalæggeren. 

Et eksempel på dette kunne være, at skolen ønskede at udbyde et større antal forskellige valgfag, men grundet skemalæggerens udfordringer, bliver alle valgfag nu nødt til at ligge på samme tidspunkt, da skemalæggeren ellers skal lave et langt større antal forskellige skemaer. Med programmet kunne skemaerne gøres mere individuelle, og skolen ville kunne udbyde flere fag og på denne måde være mere attraktiv.

\subsubsection{Skemalæggeren på skolen}
Vi vil nu kigge nærmere på en interessent, som må siges at beskæftige sig meget med problemet, nemlig skemalæggeren. I vores case lavede vi et interview med skemalæggeren på \school, hvor vi afdækkede skolens behov til skoleskemaet. I vores specifikke case, var skemaet som nævnt i afsnit \ref{Kaerbyskolen} i år blevet lagt i et samarbejde mellem alle lærerne. Dette var et forsøg på at indarbejde så mange af skolens værdier som muligt i skemaet, heriblandt de pædagogiske overvejelser i forhold til eleverne.

Mens oplevelsen med denne form for skemalægning var meget gavnende, fandt skemalæggeren hurtigt ud af, at denne måde at lægge skemaet på ikke rigtig var håndterbar i virkeligheden. Der var simpelthen for mange mennesker om processen. Måden skolen tidligere har arbejdet på, hvor \'en person lavede skemaet, havde dog nogle af de samme problemer, da denne person nemt kunne finde det uoverskueligt at tage højde for alle ønsker og opfyldte nok heller ikke så mange krav.

Ud fra vores samtale med skemalæggeren, blev det klart, at der er et ønske om at tilgodese mange behov og et tydeligt ønske fra Kærbyskolen var at fokusere på det pædagogiske element i forhold til skemalægningen, et ønske som ikke har været muligt at opfylde ved hjælp af de manuelt udarbejdede skemaer. Dette ønske om et skema med pædagogiske overvejelser udforskes nærmere i afsnit \ref{paedagogisk_laering}.

\subsubsection{Politikere}
Politikerne har igennem loven opstillet nogle krav til, hvordan timerne i skolen skal afvikles og fordeles mellem de forskellige fag. Man kunne ved hjælp af en fælles automatiseret løsning på alle landets folkeskoler fastsætte nogle fælles retningslinjer for, hvordan skemaet skal lægges. Man kunne igennem disse retningslinjer sikre sig, at landets skoler opfylder de krav, som det forventes fra politikernes side. Dog kan dette hurtigt skabe protester og andre former for klager, hvilket nemt kunne ses efter reformens indtrædelse i 2014\cite{LaererBrok}, specielt hvis de pågældende lovændringer ikke kan lade sig gøre for specifikke cases. Hvis de politikerne på Christiansborg beslutter sig for, at et kreativt fag som f.eks. billedkunst skal ligge om morgenen mellem 8:00-10:00 for alle skolernes årgange, vil dette være umuligt for f.eks. \school eftersom de låner mange faglokaler af det nærliggende Kulturskolen\cite{interview_Kaerby}. 

\subsubsection{Andre interessenter}
Vi har i denne interessentanalyse ikke taget højde for alle interessenter. Vi har blandt andet ikke være omkring eleverne og elevernes forældre. Disse har selvfølgelig også en vis interesse i, at skemaet lægges efter nogle bestemte kriterier. I forbindelse med udviklingen af et automatiseret skemalægningsprogram kan konkurrenter naturligvis også ses som interessenter. Vi har dog valgt, at vi vil fokusere på skemalæggeren, dennes opgave og hvordan vi kan optimere processen.

I den forbindelse har vi valgt at kigge på, hvilke overvejelser en skole og skemalægger gør sig i udarbejdningen af et skema, og hvilke mennesker som har indflydelse på denne proces. Vi har derfor i denne analyse fokuseret på interessenter såsom skemalæggere, da de har et direkte ansvar for skemalægningsprocessen.


%Mangler et eller andet...
%Skrammel som måske kan sættes ind et andet sted..

%I stedet for at en person skulle påtage sig opgaven for at få dette nye puslespil til at gå op, er det blot få ting, der skal ændres i et program, og et nyt skema vil være klart med det samme. Altså er man ikke længere afhængig af at en person skal kunne få de mange klasser og lærere til at spille sammen, men at et program kan finde frem til løsningen langt hurtigere. 

%Når der udbydes valgfag i folkeskolen, foregår dette ofte ved at prioritere en række fag, hvorefter skemalæggeren efter bedste evne kan forsøge at finde lærere til at undervise i disse fag. Der bliver derfor lagt et tidspunkt ind i skemaet, der blot hedder ``valgfag'', hvor eleverne så kan gå til deres respektive valgfag. Det er ikke sikkert at det er muligt at alle kommer på de valghold de havde som første prioritet, da skemaerne simpelthen kan blive umulige at få til at gå op. En automatiseret løsning vil dog ikke have disse samme begrænsninger. Den store forskel mellem et menneske og en computer i denne sammenhæng, er nemlig at mens en menneskelig skemalægger kan få skemaerne til at gå op med 10 forskellige klassetrin, kan en computer få det til at gå op med små variationer i skemaet til den enkelte elevtype (elevtype skal her forstås som en unik sammensætning af påkrævede og valgfrie fag). Der er derfor intet der stopper programmet fra at lægge nogle elevers valgfag mandag eftermiddag, mens de for andre elever i samme klasse først har valgfag torsdag morgen.

%Ved hjælp af en automatiseret løsning, som derfor løser udfordringerne ved denne personlige skemalægning, opnår skolen altså både en økonomisk fordel og et større fagligt udbud til eleverne.
