\subsection{Interessenter}
Når vi snakker om udfordringerne i forhold til skemalægningen i folkeskolen, er det ganske relevant at kigge på de grupper af mennesker som har en interesse i dette emne. Vi vil derfor i dette afsnit forsøge at klarlægge hvilke mennesker som kunne have interesse i en eventuel løsning på disse udfordringer.

\subsubsection{Skoleledelsen}

I dag skal der afsættes timer til at en eller flere ansatte på en skole, kan lægge skemaer for samtlige klasser. Ved at automatisere denne opgave ville skolen kunne spare de udgifter der opstår i forbindelse med dette. Men ikke nok med at pengene kan spares de to gange om året hvor skemaerne lægges, kan programmet bruges igen og igen, hvis der opstår ændringer i strukturen på skolen og det gamle skema viser sig at være umuligt at bruge. Hvis der for eksempel på en skole sker en lærerudskiftning, og der kommer nye lærere til skolen mens andre forlader den, er det ikke sikkert at disse lærere ville kunne undervise i de samme sammensætninger af fag som de gamle lærere.

I det konkrete eksempel ved en skole i Thisted, var der til 3 mennesker afsat 8 arbejdsdage til at få skemaet til at gå op. Lederstillinger i folkeskolen har en årsløn på mellem 500.540 kr og 729.549 kr \cite{TR_HAANDBOGEN}\cite{Statens_adm}, så 8 arbejdsdage for disse 3 stillinger bliver en enorm udgift for de 1.313 folkeskoler i Danmark\cite{UVM-Folkeskoler}. Nogle af disse 8 arbejdsdage blev også brugt på seminarer, hvilket man kunne forestille sig stadig ville være nødvendigt, hvis ikke der skulle lægges skemaer. Men de dage disse ledere skulle tilbringe på skolen med at lave skemaer, kunne nu bruges på andre gøremål i stedet, eller helt undværes.

Ved at automatisere løsningen, ville der forhåbentligt kunne opnås en større fleksibilitet i skemaet, da et computersystem har muligheden for at forsøge mange flere skemaer end en person. Ved at skemaerne kan genereres hurtigere ved hjælp af en computer, er det også muligt at indarbejde flere ønsker og begrænsninger i skemaet, uden at det bliver en uoverskueligt for skemalæggeren. 

Et eksempel på dette kunne være at skolen ønskede at udbyde et større antal af forskellige valgfag, men grundet skemalæggerens udfordringer, bliver alle valgfag til at ligge på samme tidspunkt, da skemalæggeren ellers skal lave et langt større antal forskellige skemaer. Med programmet kunne skemaerne gøres mere individuelle, og skolen ville kunne udbyde flere fag, og på denne måde være mere attraktiv at vælge, frem for andre.

\subsubsection{Skemalæggeren på skolen}
I vores case lavede vi et interview med skemalæggeren på \school hvor vi afdækkede skolens behov til skoleskemaet. I vores specfikke case, var skemaet i år blevet lagt i et samarbejde mellem alle lærerne. Dette var et forsøg på at indarbejde flest muligt pædagogiske overvejelser i skemaet, og altså indarbejde flest mulige overvejelser i skolen.

Mens oplevelsen med denne form for skemalægning var meget gavnende, fandt skemalæggeren hurtigt ud af, at denne måde at lægge skemaet på ikke var håndterbar i virkeligheden. Der var simpelthen for mange mennesker om processen. Tidligere havde skolen arbejdet på en sådan måde, at der var \'en person som var ansvarlig for at lægge skemaet, her var dog det modsatte problem, med at det var svært ene mand at lave et skema som tilgodeså alle lærerer samt elever.

Ud fra vores samtale med skemalæggeren, blev det klart at der er et ønske om at tilgodese mange behov, og et tydeligt ønske fra Kærbyskolen var at fokusere på det pædagogiske element i forhold til skemalægningen, et ønske som ikke har været muligt at opfylde ved hjælp af de manuelt udarbejde skemaer. Dette ønske om et skema med pædagogiske overvejelser, udforskes nærmere i afsnit \ref{paedagogisk_laering}.



\subsubsection{Politikere}
Politikerne har igennem lovet opstillet nogle krav til hvordan timerne i skolen skal fordeles mellem de forskellige fag. Man kunne ved hjælp af en fælles automatiseret løsning på alle landets folkeskoler fastsætte nogle fælles retningslinjer for hvordan skemaet skal lægges. Man kunne igennem disse retningslinjer sikre sig at landets skoler opfylder de krav som det forventes fra politikernes side. 

%Mangler et eller andet...
%Skrammel som måske kan sættes ind et andet sted..
\iffalse
I stedet for at en person ville skulle påtage sig opgaven at få dette nye puslespil til at gå op, er det blot få ting der skal ændres i et program, og et nyt skema vil være klart med det samme. Altså er man ikke længere afhængig af at en person skal kunne få de mange klasser og lærere til at spille sammen, men at et program kan finde frem til løsningen langt hurtigere. 

Når der udbydes valgfag i folkeskolen, foregår dette ofte ved at prioritere en række fag, hvorefter skemalæggeren efter bedste evne kan forsøge at finde lærere til at undervise i disse fag. Der bliver derfor lagt et tidspunkt ind i skemaet der blot hedder ``valgfag'', hvor eleverne så kan gå til deres respektive valgfag. Det er ikke sikkert at det er muligt at alle kommer på de valghold de havde som første prioritet, da skemaerne simpelthen kan blive umulige at få til at gå op. En automatiseret løsning vil dog ikke have disse samme begrænsninger. Den store forskel mellem et menneske og en computer i denne sammenhæng er nemlig, at mens en menneskelig skemalægger kan få skemaerne til at gå op med 10 forskellige klassetrin, kan en computer få det til at gå op med små variationer i skemaet til den enkelte elevtype (elevtype skal her forstås som en unik sammensætning af påkrævede og valgfrie fag). Der er derfor intet der stopper programmet fra at lægge nogle elevers valgfag mandag eftermiddag, mens de for andre elever i samme klasse først har valgfag torsdag morgen.

Ved hjælp af en automatiseret løsning, som derfor løser udfordringerne ved denne personlige skemalægning, opnår skolen altså både en økonomisk fordel og et større fagligt udbud til eleverne.
\fi