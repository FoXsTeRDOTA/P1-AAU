\subsection{Pædagogisk læring}
\label{paedagogisk_laering}
På \school nævnte skemalæggeren et ønske om et skema udarbejdet med pædagogiske overvejelser i fokus. Altså et skema som tilgodeser elevernes behov for at opnå den højest mulige indlæring.

Da skemaet i år blev udarbejdet manuelt, fandt Kærbyskolen det umuligt at lave et skema, der tilgodeser alle de pædagogiske overvejelser, de havde gjort sig for samtlige klasser\cite{interview_Kaerby}.

På Tingstrup skole sikrede de sig, at ingen klasser i indskolingen (0.-3. klassetrin) eller mellemtrinnet (4.-6. klassetrin) havde dage bestående udelukkende af kreative fag eller dage bestående udelukkende af boglige fag. 

Undervisningsministeriet tilsidesætter 1 milliard kroner i perioden 2014-2020 til at styrke udviklingen af pædagogiske værktøjer i folkeskoler. Læringskonsulenterne vil vejlede skoler i de ting, som gør en skoledag mere varieret, selvom dagene bliver længere. Skoler vil få vejledning om, hvordan mulighederne, som en længere skoledag giver, kan bruges til at få mest ud af elevernes læring. Nogle af løsningerne læringskonsulenterne kunne komme med er:
\begin{itemize}
	\item At blande formidling af teori med praktiske arbejdsformer.
	\item At give mulgihed for, at eleverne kan bringe et bredere udsnit af deres evner og interesser i spil.
	\item At have mere bevægelse som en naturlig del af faglige aktiviteter\cite{Paedagogisklaering}.
\end{itemize}