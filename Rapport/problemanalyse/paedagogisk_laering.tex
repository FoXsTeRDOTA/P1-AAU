\subsection{Pædagogisk læring}
\label{paedagogisk_laering}
På \school nævnte skemalæggeren et ønske om et skema udarbejdet med pædagogiske overvejelser i fokus. Altså et skema som på bedste vis tilgodeser elevernes behov for at opnå den højest mulige indlæring.

Da skemaet i år blev udarbejdet manuelt, fandt Kærbyskolen det umuligt at lave et skema der tilgodeser alle de pædagogiske overvejelser de havde gjort sig, for samtlige klasser\cite{interview_Kaerby}.

På Tingstrup skole sikrede de sig, at ingen klasser i indskolingen (0.-3. klassetrin) eller mellemtrinnet (4.-6. klassetrin) havde dage bestående udelukkende af kreative fag eller dage bestående udelukkende af boglige fag. 

Undervisningsministerie sætter 1 milliard kroner til siden i perioden 2014-2020 for at styrke udviklingen af pædagogiske værktøjer i folkeskoler. Læringskonsulenterne vil vejledeskoler i de ting som gør en skoledag mere varieret selvom dagene bliver længere. Skoler vil få vejledning om, hvordan mulighederne en længere skoledag giver, kan bruges for at få mest ud af elevernes læring. Nogle af løsningerne læringskonsulenterne kunne komme med er:
\begin{itemize}
\item At blande formidling af teori med praktiske arbejdsformer.
\item At giv mulgihed for at eleverne kan bringe et breder udsnit af deres evner og interesser i spil.
\item At indeholde mere bevægelse som en naturlig del af faglige aktiviteter\cite{Paedagogisklaering}.

\end{itemize}