
I vores problemafgrænsning ønsker vi at indsnævre det problem, som tages med fra problemanalysen og finde frem til de fokusområder, som vi finder vigtigst inden for skemalægning i folkeskoler og ønsker ar tage udgangspunkt i under problemløsningen. Prolemafgrænsning skulle altså gerne hjælpe med at finde frem til en god problemformulering. Det vil i afsnittet altså blive beskrevet hvilke fokusområder der til- og fravælges.

For at sikre et projekt, som har relevans i vores tid og for vores specifikke case, tages der udgangspunkt i vores case, Kærbyskolen.

\subsection*{Pædagogisk læring}
På Kærbyskolen har skemaer, som tager deres udgangspunkt i pædagogisk læring og elevtrivsel i år været højt prioriteret. Dette kom til udtryk under interview med skolen, hvor det blev beskrevet, at pædagogiske overvejelser i forhold til skemalægning blev diskuteret meget og var noget, som nuværende løsninger i følge skolen ikke forholder sig nok til. Vi ønsker derfor at lægge fokus på skemalægning med vægt på pædagogisk overvejelser.

\subsubsection*{Brugergrænseflade}
Brugergrænseflade fravælges som fokusområde og dette gøres hovedsageligt grundet projektets omfang og tidsbegrænsning. Dette beytder dog ikke, at brugervenlighed glemmes helt. Kærbyskolen lagde under det foretagede interview vægt på, at programmer skal være nemme at bruge, og at det er vigtigt, at programmet løbende giver brugeren en status. Dette ønsker vi så at tage hensyn til i så stor et omfang, som det er muligt.

\subsubsection*{Andre afgrænsninger}
I forbindelse med overvejelser omkring projektets omfang og de opstillede tidsbegrænsninger, som vi må forholde os til, har vi desuden måtte gøre os nogle andre fravalg af fokuspunkter. Dette er valg som vi føler er nødvendige for, at projektet realistisk kan gennemføres i tide og samtidig ikke rammer den egentlige kvalitet af problemløsningen eller programfunktionalitet, men blot gør det endelige program lidt mindre.

Vi har med valget af Kærbyskolen som case fravalgt at tage højde for valgfag, da skolen kun går til 6. klasse, hvor disse endnu ikke findes. 

Vores case, Kærbyskolen, er en meget lille skole og derfor  har de ikke mulighed for at have alle fagene på skolensgrund. Kærbyskolen er nødt til at arbejde sammen med andre parter for at opfylde kraverne om idræts og hjemkundsskabsfaciliteter. Vi har dog valgt, at fokuset ikke skal være lagt på lokalerne. Men på pædagogisk læring og brugerflade.

Vi fokuserer på skemalægning på Kærbyskolen, men ønsker ikke at tage højde for skolens såkaldte J-klasser for autismediagnostiserede elever, da der dannes skemaer efter elevens individuelle behov. Derfor vil vi fokusere på at lægge skema til klasserne 0 til 6.

%\subsubsection*{Valgfag}
%Vi har fravalgt valgfag, som et fokusemne, da vi har valgt at arbejde ud fra en case, hvor skolen i dette tilfælde går op til 6. klasse, hvor de ikke har valgfag.

%\subsubsection*{Lokaler}
%Vores case, Kærbyskolen, er en meget lille skole og derfor  har de ikke mulighed for at have alle fagene på skolensgrund. Kærbyskolen er nødt til at arbejde sammen med andre parter for at opfylde kraverne om idræts og hjemkundsskabs faciliteter. Vi har dog valgt, at fokuset ikke skal være lagt på lokalerne, men på pædagogisk læring og brugerflade.

%Vi fokuserer på skemalægning på Kærbyskolen, men ønsker ikke at tage højde for skolens såkaldte J-klasser for autismediagnostiserede elever, da der dannes skemaer efter elevens individuelle behov. Derfor vil vi fokusere på at lægge skema til klasserne 0 til 6.
