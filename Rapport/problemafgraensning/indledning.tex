
I vores problemafgrænsning vil vi indsnævre problemet fra problemanalysen og finde frem til fokuspunkter, som vi vil uddybe og finde frem til en problemformulering.

\subsection*{Fokusområder}
For at finde frem til fokusområdet, har vi valgt at være selektive med problemerne, der opstår omkring skemalægning i skoler. Vi tager udgangspunkt i vores case med Kærbyskolen, hvor de gerne vil lægge meget vægt på design af programmet, så det skulle være nemt at bruge, og skemaet skulle lægge fokus på pædagogisk læring. 

Vi fokuserer på skemalægning på Kærbyskolen, men vi kan ikke tage højde for de speciale klasser på skolen, da de danner skemaer efter individuelle behov. Derfor vil vi fokusere på at lægge skema til klasserne 0 til 6.

\subsubsection*{Pædagogisk læring}
Vi har valgt at fokusere på pædagogisk læring, da det er noget skolen vægter meget i forhold til undervisningen. Skolen mener, at det kunne være noget som skiller dem ud fra andre skoler.
(Mere skal skrives)

\subsubsection*{Brugerflade}
Valget af brugerflade skyldes, at lærerne ønsker et system, som er brugervenlige og pænt at kigge på, så som funktionelt. Dette har vi også valgt at fokusere på, da vi ud fra vores interview med skolen, har kommet frem til at det er noget som lærene ville sætte stort pris på.

\subsubsection*{Valgfag}
Vi har fravalgt valgfag, som et fokusemne, da vi har valgt at arbejde ud fra en case, hvor skolen i dette tilfælde går op til 6. klasse, hvor de ikke har valgfag.

\subsubsection*{Lokaler}
Vores case, Kærbyskolen, er en meget lille skole og derfor  har de ikke mulighed for at have alle fagene på skolensgrund. Kærbyskolen er nødt til at arbejde sammen med andre parter for at opfylde kraverne om idræts og hjemkundsskabs faciliteter. Vi har dog valgt, at fokuset ikke skal være lagt på lokalerne. Men på pædagogisk læring og brugerflade.

