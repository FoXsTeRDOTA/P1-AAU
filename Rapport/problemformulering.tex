På baggrund af de observationer og erfaringer, der er blevet gjort gennem rapportens problemanalyse, er problemet i afsnit \ref{afg} blevet afgrænset, og vi er kommet nærmere det, som vi mener er kernen i vores problemstilling om skemalægning i danske folkeskoler. Vi er nået frem til, at det der er behov for på folkeskoler ikke blot er en nemmere, hurtigere form for skemalægning, men også en skemalægning, som har fokus på pædagogiske overvejelser. Dette har givet os nedenstående problemformulering.

``Hvordan hjælper man, ved hjælp af software, skemalæggeren med at lægge vægt på pædagogiske overvejelser under skemalægingen?''