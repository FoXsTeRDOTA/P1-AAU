\documentclass[12pt,hidelinks]{article}
\usepackage{graphicx,hyperref,amsmath,bm,url}
\usepackage[numbers]{natbib}
\usepackage{microtype,todonotes}
\usepackage{a4}
\usepackage[compact,small]{titlesec}
\usepackage[utf8]{inputenc}
\usepackage{placeins}
\clubpenalty = 10000
\widowpenalty = 10000
\usepackage[T1]{fontenc}
\graphicspath{ {../Billeder/} }
\usepackage{tikz}
\usetikzlibrary{calc}
\usetikzlibrary{shapes}
\usepackage[labelfont=bf]{caption}
\renewcommand{\figurename}{\textbf{Figur}}
\renewcommand{\contentsname}{Indholdfortegnelse}
\usepackage[nottoc,notlof,notlot]{tocbibind} 
\renewcommand\bibname{Referencer}
\renewcommand{\tablename}{Tabel}


\makeatletter
\newdimen\@myBoxHeight%
\newdimen\@myBoxDepth%
\newdimen\@myBoxWidth%
\newdimen\@myBoxSize%
\newcommand{\SquareBox}[2][]{%
    \settoheight{\@myBoxHeight}{#2}% Record height of box
    \settodepth{\@myBoxDepth}{#2}% Record depth of box
    \settowidth{\@myBoxWidth}{#2}% Record width of box
    \pgfmathsetlength{\@myBoxSize}{max(\@myBoxWidth,(\@myBoxHeight+\@myBoxDepth))}%
    \tikz \node [shape=rectangle, shape aspect=1,draw=red,inner sep=2\pgflinewidth, minimum size=\@myBoxSize,#1] {#2};%
}%
\makeatother
\newcommand*{\captionsource}[2]{%
  \caption[{#1}]{%
    #1%
    \\\hspace{\linewidth}%
    \textbf{Kilde:} #2%
  }%
}

\newcommand{\school}{Kærbyskolen }
\setcounter{secnumdepth}{4}

\begin{document}
	\sloppy

	\input{titlepage.tex}

	\newpage
	\section*{Forord}
	Denne rapport er skrevet af gruppe DAT1 B2-2 på Datalogistudiet på Aalborg Universitet i forbindelse med studiets P1-projekt. Forløbets titel er "Fra eksisterende software til modeller". Projektet skal afspejle de færdigheder og kompentencer, der stilles som krav, herunder at kunne forstå og gøre rede for anvendte teorier og metoder til analyse og anvendelse af datalogiske begreber inden for programmering samt modellering. Gruppen har i rapporten taget udgangspunkt i Kærbyskolen i Aalborg, som der gennem hele forløbet har været et tæt samarbejde med.

Projektgruppen vil i denne anledning gerne rette en stor tak til følgende institutioner og personer for medvirken i projektet i form af interviews mm.:

Kærbyskolen, ledelsesrepræsentant Jesper Foldberg Nielsen.
	\newpage
	\tableofcontents
	\newpage
	\section{Indledning}
	
I vores problemafgrænsning ønsker vi at indsnævre det problem, som tages med fra problemanalysen og finde frem til de fokusområder, som vi finder vigtigst inden for skemalægning i folkeskoler og ønsker ar tage udgangspunkt i under problemløsningen. Prolemafgrænsning skulle altså gerne hjælpe med at finde frem til en god problemformulering. Det vil i afsnittet altså blive beskrevet hvilke fokusområder der til- og fravælges.

For at sikre et projekt, som har relevans i vores tid og for vores specifikke case, tages der udgangspunkt i vores case, Kærbyskolen.

\subsection*{Pædagogisk læring}
På Kærbyskolen har skemaer, som tager deres udgangspunkt i pædagogisk læring og elevtrivsel i år været højt prioriteret. Dette kom til udtryk under interview med skolen, hvor det blev beskrevet, at pædagogiske overvejelser i forhold til skemalægning blev diskuteret meget og var noget, som nuværende løsninger i følge skolen ikke forholder sig nok til. Vi ønsker derfor at lægge fokus på skemalægning med vægt på pædagogisk overvejelser.

\subsubsection*{Brugergrænseflade}
Brugergrænseflade fravælges som fokusområde og dette gøres hovedsageligt grundet projektets omfang og tidsbegrænsning. Dette beytder dog ikke, at brugervenlighed glemmes helt. Kærbyskolen lagde under det foretagede interview vægt på, at programmer skal være nemme at bruge, og at det er vigtigt, at programmet løbende giver brugeren en status. Dette ønsker vi så at tage hensyn til i så stor et omfang, som det er muligt.

\subsubsection*{Andre afgrænsninger}
I forbindelse med overvejelser omkring projektets omfang og de opstillede tidsbegrænsninger, som vi må forholde os til, har vi desuden måtte gøre os nogle andre fravalg af fokuspunkter. Dette er valg som vi føler er nødvendige for, at projektet realistisk kan gennemføres i tide og samtidig ikke rammer den egentlige kvalitet af problemløsningen eller programfunktionalitet, men blot gør det endelige program lidt mindre.

Vi har med valget af Kærbyskolen som case fravalgt at tage højde for valgfag, da skolen kun går til 6. klasse, hvor disse endnu ikke findes. 

Vores case, Kærbyskolen, er en meget lille skole og derfor  har de ikke mulighed for at have alle fagene på skolensgrund. Kærbyskolen er nødt til at arbejde sammen med andre parter for at opfylde kraverne om idræts og hjemkundsskabsfaciliteter. Vi har dog valgt, at fokuset ikke skal være lagt på lokalerne. Men på pædagogisk læring og brugerflade.

Vi fokuserer på skemalægning på Kærbyskolen, men ønsker ikke at tage højde for skolens såkaldte J-klasser for autismediagnostiserede elever, da der dannes skemaer efter elevens individuelle behov. Derfor vil vi fokusere på at lægge skema til klasserne 0 til 6.

%\subsubsection*{Valgfag}
%Vi har fravalgt valgfag, som et fokusemne, da vi har valgt at arbejde ud fra en case, hvor skolen i dette tilfælde går op til 6. klasse, hvor de ikke har valgfag.

%\subsubsection*{Lokaler}
%Vores case, Kærbyskolen, er en meget lille skole og derfor  har de ikke mulighed for at have alle fagene på skolensgrund. Kærbyskolen er nødt til at arbejde sammen med andre parter for at opfylde kraverne om idræts og hjemkundsskabs faciliteter. Vi har dog valgt, at fokuset ikke skal være lagt på lokalerne, men på pædagogisk læring og brugerflade.

%Vi fokuserer på skemalægning på Kærbyskolen, men ønsker ikke at tage højde for skolens såkaldte J-klasser for autismediagnostiserede elever, da der dannes skemaer efter elevens individuelle behov. Derfor vil vi fokusere på at lægge skema til klasserne 0 til 6.

	\section{Problemanalyse}
	I det følgende kommer vi ind under bestemte udfordringer som folkeskolen står over for, når der lægges skemaer. Dette er blandt andet lovgivningen. Vi kommer derudover også ind på en specifik skoles udfordringer samt diverse interessenter.
	\subsection{Lovgivning og regler for skemalægning}
\label{Lovgivning og regler}
I driften af en folkeskole er planlægning et vigtigt redskab i opnåelsen af en velfungerende og læringsrig hverdag for såvel lærere som elever, sekretærer og andet personale. Planlægning er afgørende og et godt skema ligeså. I folkeskolen er hverdagen som hovedregel baseret på de ugeskemaer, der som minimum inden starten af hvert skoleår udarbejdes af skolen og sidenhen følges nøje.\cite{interview2} Fungerer dette skema ikke, er der risiko for, at hverdagen ej heller fungerer og undervisningsniveauet rammes af det.

Dette er en af grundene til, at der hvert år lægges mange timer i skemalægning på hver eneste folkeskole i landet og oftest er der tale om et større team, som må samle sig om opgaven. Hvad der gør det så svært at lægge et velfungerende skema er den nærmeste uendelige liste af lovgivninger, regler, krav og bindinger som de skemalæggende teams er nødsagede til at tage hensyn til.  Der stilles krav til skolen af det danske undervisningsministerium i form af bl.a. de nationale Fælles Mål \cite{fmaal}, som skal være styrende for undervisningen og er mål for, hvad eleverne skal lære i de enkelte fag, men i særdeleshed folkeskolereformen og bekendtgørelsen af lov om folkeskolen \cite{Lovgivning} er styrende for skemalægningsprocessen. Under denne bekendtgørelse findes blandt andet en offentliggørelse af folkeskolernes minimums- og vejledende timetal (Bilag 2) for de enkelte fag og årgange, som skal følges. Dette giver skemalæggeren et udgangspunkt, men binder samtidig og gør processen mindre fleksibel. Desuden må der tages hensyn til lærernes arbejdstider, tid til forberedelse, skolens egne praktiske og pædagogiske krav og ønsker og i særdeles mange tilfælde, udefrakommende bindinger i form af lån af speciallokaler og så videre.

% Please add the following required packages to your document preamble:
%\usepackage{graphicx}
\iffalse
\begin{table}[]
	\centering
	\caption{My caption}
	\label{TimetalsKrav}
	\resizebox{\textwidth}{!}{%
		\begin{tabular}{lllllllllllll}
			Timetal (minimumstimetal og vejledende timetal) for fagene &                               &     &     &     &     &     &     &     &     &     &     &                         \\
			Klassetrin                                                 &                               & Bh. & 1.  & 2.  & 3.  & 4.  & 5.  & 6.  & 7.  & 8.  & 9.  & Timetal i alt           \\
			\textbf{A. Humanistiske fag}                               &                               &     &     &     &     &     &     &     &     &     &     &                         \\
			Dansk                                                      & \textit{(minimumstimetal)}    &     & 330 & 300 & 270 & 210 & 210 & 210 & 210 & 210 & 210 & 2.160                   \\
			Engelsk                                                    & \textit{(vejledende timetal)} &     & 30  & 30  & 60  & 60  & 90  & 90  & 90  & 90  & 90  & 630                     \\
			Tysk eller fransk                                          & \textit{(vejledende timetal)} &     &     &     &     &     & 30  & 60  & 90  & 90  & 90  & 360                     \\
			Historie                                                   & \textit{(minimumstimetal)}    &     &     &     & 30  & 60  & 60  & 60  & 60  & 60  & 30  & 360                     \\
			Kristendomskundskab                                        & \textit{(vejledende timetal)} &     & 60  & 30  & 30  & 30  & 30  & 60  &     & 30  & 30  & 300                     \\
			Samfundsfag                                                & \textit{(vejledende timetal)} &     &     &     &     &     &     &     &     & 60  & 60  & 120                     \\
			\textbf{B. Naturfag}                                       & \textit{}                     &     &     &     &     &     &     &     &     &     &     &                         \\
			Matematik                                                  & \textit{(minimumstimetal)}    &     & 150 & 150 & 150 & 150 & 150 & 150 & 150 & 150 & 150 & 1.350                   \\
			Natur/teknik                                               & \textit{(vejledende timetal)} &     & 30  & 60  & 60  & 90  & 60  & 60  &     &     &     & 360                     \\
			Geografi                                                   & \textit{(vejledende timetal)} &     &     &     &     &     &     &     & 60  & 30  & 30  & 120                     \\
			Biologi                                                    & \textit{(vejledende timetal)} &     &     &     &     &     &     &     & 60  & 60  & 30  & 150                     \\
			Fysik/kemi                                                 & \textit{(vejledende timetal)} &     &     &     &     &     &     &     & 60  & 60  & 90  & 210                     \\
			&                               &     &     &     &     &     &     &     &     &     &     &                         \\
			\textbf{C. Praktiske/musiske fag}                          &                               &     &     &     &     &     &     &     &     &     &     &                         \\
			Idræt                                                      & \textit{(vejledende timetal)} &     & 60  & 60  & 60  & 90  & 90  & 90  & 60  & 60  & 60  & 630                     \\
			Musik                                                      & \textit{(vejledende timetal)} &     & 60  & 60  & 60  & 60  & 60  & 30  &     &     &     & 330                     \\
			Billedkunst                                                & \textit{(vejledende timetal)} &     & 30  & 60  & 60  & 60  & 30  &     &     &     &     & 240                     \\
			Håndværk og design samt madkundskab                        & \textit{(vejledende timetal)} &     &     &     &     & 90  & 120 & 120 & 60  &     &     & 390                     \\
			\textbf{D. Valgfag}                                        & \textit{(vejledende timetal)} &     &     &     &     &     &     &     & 60  & 60  & 60  & 180                     \\
			\textbf{E. Årligt minimumstimetal pr. klassetrin}          &                               & 600 & 750 & 750 & 780 & 900 & 930 & 930 & 960 & 960 & 930 & 7.890 ekskl. bh. /8.490
		\end{tabular}
	}
\end{table}
Note: Timetallene er angivet i klokketimer og uden pauser.Note: Bh.: Børnehaveklasse.
\fi
	\subsection{Reformen}
\label{Reformen}
Folkeskolerne i Danmark, har igennem de sidste par år gennemgået en stor forandring. Den danske regering har valgt at ændre systemet for at sørge for, at eleverne får størst mulig udbytte af deres undervisning uanset hvilken baggrund de har, eller hvor fagligt stærke de er.
Dem der står bag denne reform opstiller få klare mål:
	\begin{itemize}
		\item Folkeskolen skal udfordre alle elever, så de bliver så dygtige, de kan.
		\item Folkeskolen skal mindske betydningen af social baggrund for de faglige resultater.
		\item Tilliden til og trivslen i folkeskolen skal styrkes gennem blandt andet respekt for professionel viden og praksis.
	\end{itemize}

Reformen trådte i kraft fra skolestarten i 2014. Siden denne skolestart har eleverne fra folkeskolerne haft en længere skoledag end de plejede inden reformen. Den ekstra tid i skolen eleverne får betyder, at der er mere tid til, at den enkelte elev kan lære mere. Hvor meget ekstra tid hver elev får afhænger af elevens alder. For eksempel slutter de mindste elever typisk kl. 14 og de ældste omkring kl 15.

Man får også flere timer til to såkaldte ”kernefag”, matematik og dansk, som ses som værende grundlæggende for at kunne forstå og effektivt lære indenfor andre fagområder. Udover flere timer til de mest grundlæggende, meget faglige fag skal motion også implementeres i elevernes skoledag med i gennemsnit 45 minutter daglig bevægelse. Udover den ekstra tid til mere undervisning, kommer der også en obligatorisk lektiecafé, hvilket vil indgå i elevernes skemaer. Denne lektiecafé har til måls at sikre, at eleverne får lavet deres lektier og får størst muligt udbytte af at lave disse.

Der kommer også en stor investering i optimering af undervisningen. Undervisningen skal være af en højere kvalitet end forinden reformen og kompetencen af lærere og pædagoger skal øges, så de kan varetage de krav og mål som folkeskolereformen sætter. Undervisningsministeriet siger, at alle lærere skal have undervisningskompetence i fagene de underviser i, og der er afsat ca. 1 milliard kroner fra 2014-2020 til, at lærerne kan få en styrket efteruddannelse. Det vil resultere i, at eleverne kan få mere tid sammen med bedre kvalificerede lærere.

Med indføringen af reformen bliver der også indført nogle regelforenklinger af folksekolelove, så kommunerne får mere frihed til at lave undervisning efter det lokale miljø. Det vil sige, at der kommer flere muligheder for fleksible regler for en skolebestyrelse.

	\subsection{Generelt ved skemalægning}
Eftersom det er lovpligtigt for de offentlige folkeskoler at følge skolereformen, da denne går ind under den danske lovgivning, skal alle landets skoler designe et skema for alle skolens årgange og klasser, som stemmer overens med blandt andet minimumstimetal- og vejledende-timetalskravende som er vist i tabel \ref{TimetalsKrav}. Eftersom der er 200 skoledage om året i en folkeskole\cite{elevers_timetal}, ved hver skole hvor mange timer de ugenligt, skal sætte af til dansk, matematik, osv. 


%Alle skoler lægger deres skemaer ud fra restriktioner så som dette, og en af de skoler som gør netop dette, er \school (bilag \ref{InterviewKaerby}), som har mange restriktioner pga. det er en lille skole og derfor låner mange af deres specialelokaler så som musiklokaler og idrætfacilliter fra Kulturhuset. Skolen her havde allerede lagt disse forskellige lektioner ind i skemaerne de skulle til at lægge før de overhovedet gik i gang med at fylde skemaet ud, ud fra minimum og vejledende timetalskravene. Disse special timer var vigtige at placere da de havde bestemte restriktioner, der gjorde at hvis man skulle have et udbytte af timen ville det være nædvendigt at ligge dem på tidspunkter hvor de vil have lokaler til det. Så havde man på \school valgt at give opgaven til lærerne selv. Så de delt op i teams, og lavede derfra et skema til de forskellige årgange som de var undervisere for (bilag \ref{InterviewKaerby}). Her var alle skolens ansatte med inde under selve skemalægnings processen. 

%En anden skole vi har interviewet (bilag \ref{InterviewTingstrup}), havde ikke de samme restriktioner indenfor lokaler, så de begyndte at ligge skemaerne ud fra timetalskravende. Ligesom ved \school, lagde Tingstrup også skemaerne sammen i grupper, dog en væsentlig mindre gruppe (tre frem for treds). En anden meget væsentlig forskel mellem disse 2 skoler er, at de på Tingstrup stod 3 ledere for skemalægningen, hvor de på \school havde alle ansatte med til skemalægningen. Vi kan ud fra disse to cases, godt regne med at skemalægningen oftest ikke er noget som kun en person står med.

%Tidligere havde \school brugt programmet Docendo og Tingstrup havde tidligere brugt Matrix, til at lægge deres skemaer i. Begge skoler ligger deres skemaer ud på Skoleintra så både lærere, elever og forældre, kan se hvilke timer de har og hvornår. Det er en væsentlig ting at både lærer og elever let har adgang til deres dagsorden og for elevernes vedkommende, også deres lektier og afleveringsopgaver, så uanset om skolerne laver deres skema manuelt, eller lægger dem i et stykke software, er det en generalt ting at ligge skemaerne op på en platform som Skoleintra.

	\subsection{Kaerbyskolen}
\label{Kaerbyskolen}
Vi har interviewet en folkeskole i Aalborg kommune, Kærbyskole, for at kunne danne et skema, som vil opfylde behovene hos den ene skole. Men for at finde ud af hvad andre skoler bruger af systemer og hvad de ville finde fornuftigt at systemet indeholder og hvad kunne bruges for at optimere deres nuværende system, fik vi også interviewet Tingstrup skole i Thisted.

Vores fokus ligger ved Kærbyskolen i Aalborg.

Skolen består af omkring 75 medarbejder, som er uddelt blandt ledelse, børnehaveklasseledere, lærere, pædagoger, teknisk administrativt personale og rengøringspersonale. Ud af de 75 medarbejder er 33 af dem lære.

Kærbyskolen går fra børnehaveklassen til 6. klasse. Og på hver klassetrin har de to klasser. Udover det har Kærbyskolen fire specialklasser for børn og unge inden for autisme, som består af 32 elever. Skemaer til resten af skolen, udarbejdes normalt en gang pr. år.Tilsammen har skolen omkring 335 elever.

En del af skolens undervisning foregår udenfor skolens områder, da de mangler lokaler til speciale fag, som kræver lidt mere end et almindeligt klasseværelse, som idræt-, hjemkundskab- og sløjtfaciliter. Skolen har også en aftale med den lokale kulturskole, hvor de har mulighed for at låne lokaler efter aftale med kulturskolen.
\subsubsection{Skemaer}
\label{Skemaer}
Børnene i specialklasserne får en individuel struktureret undervisning og får udarbejdet en undervisningsplan tilrettet til hvert barns behov. %http://www.kaerbyskolen.skoleintra.dk/Infoweb/Indhold/Skoleplan/Skoleplan2011/j-klasser/faglig%20udvikling.htm
Skemaer til de andre elever, udarbejdes normalt en gang pr. år.


\subsubsection{Reformens indflydelse}
\label{Reformens_indflydelse}
Kærbyskolen sigter efter at følge de mål skolereformen har introduceret pr. 1 august 2014. Ledelsen af Kærbyskole har bestemt, at hverdagen på skolen skal være en varieret og spændende skoledag.

Skolereformen betyder, at de nye skemaer skal tage højde for en længere skoledag. I denne nye længde af dage, skal den nye skoledag have tid til flere timer i grundlæggende fag, som dansk og matematik, plus tid til mere motion. Fremmedsprog, skal også sættes ind i skemaet allerede fra 1. klasse. Der er også kommet obligatoriske lektiecafeer, hvor eleverne kan få hjælp fra lærerne. Lektiecafén bliver implenteret, som et normalt fag og vil normalt komme til at ligge sidst på dagen på grund af aftalen med kulturskolen.
%{http://kaerbyskolen.skoleporten.dk/sp/118224/iframe?address=http%253a%252f%252fwww.kaerbyskolen.skoleintra.dk%252fInfoweb%252fIndhold%252fSkolereform%252fSkolereform.htm}


	\subsection{Interessenter}
Når vi snakker om udfordringerne i forhold til skemalægningen i folkeskolen, er det ganske relevant at kigge på de grupper af mennesker som har en interesse i dette emne. Vi vil derfor i dette afsnit forsøge at klarlægge hvilke mennesker som kunne have interesse i en eventuel løsning på disse udfordringer.

\subsubsection{Skoleledelsen}

I dag skal der afsættes timer til at en eller flere ansatte på en skole, kan lægge skemaer for samtlige klasser. Ved at automatisere denne opgave ville skolen kunne spare de udgifter der opstår i forbindelse med dette. Men ikke nok med at pengene kan spares de to gange om året hvor skemaerne lægges, kan programmet bruges igen og igen, hvis der opstår ændringer i strukturen på skolen og det gamle skema viser sig at være umuligt at bruge. Hvis der for eksempel på en skole sker en lærerudskiftning, og der kommer nye lærere til skolen mens andre forlader den, er det ikke sikkert at disse lærere ville kunne undervise i de samme sammensætninger af fag som de gamle lærere.

I det konkrete eksempel ved en skole i Thisted, var der til 3 mennesker afsat 8 arbejdsdage til at få skemaet til at gå op. Lederstillinger i folkeskolen har en årsløn på mellem 500.540 kr og 729.549 kr \cite{Statens_adm}\cite{TR_HAANDBOGEN}, så 8 arbejdsdage for disse 3 stillinger bliver en enorm udgift for de 1.313 folkeskoler i Danmark\cite{UVM-Folkeskoler}. Nogle af disse 8 arbejdsdage blev også brugt på seminarer, hvilket man kunne forestille sig stadig ville være nødvendigt, hvis ikke der skulle lægges skemaer. Men de dage disse ledere skulle tilbringe på skolen med at lave skemaer, kunne nu bruges på andre gøremål i stedet, eller helt undværes.

Ved at automatisere løsningen, ville der forhåbentligt kunne opnås en større fleksibilitet i skemaet, da et computersystem har muligheden for at teste mange flere skemaer end en person. Ved at skemaerne kan genereres hurtigere ved hjælp af en computer, er det også muligt at indarbejde flere ønsker og begrænsninger i skemaet, uden at det bliver en uoverskueligt for skemalæggeren. 

Et eksempel på dette kunne være at skolen ønskede at udbyde et større antal af forskellige valgfag, men grundet skemalæggerens udfordringer, bliver alle valgfag til at ligge på samme tidspunkt, da skemalæggeren ellers skal lave et langt større antal forskellige skemaer. Med programmet kunne skemaerne gøres mere individuelle, og skolen ville kunne udbyde flere fag, og på denne måde være mere attraktiv at vælge, frem for andre.

\subsubsection{Skemalæggeren på skolen}
Vi vil nu kigge nærmere på en interessent, som må siges at beskæftige sig meget med problemet, nemlig skemalæggeren. I vores case lavede vi et interview med skemalæggeren på \school hvor vi afdækkede skolens behov til skoleskemaet. I vores specfikke case, var skemaet i år blevet lagt i et samarbejde mellem alle lærerne. Dette var et forsøg på at indarbejde flest muligt at skolens værdier i skemaet, heriblandt de pædagogiske overvejelser i forhold til eleverne.

Mens oplevelsen med denne form for skemalægning var meget gavnende, fandt skemalæggeren hurtigt ud af, at denne måde at lægge skemaet på ikke var håndterbar i virkeligheden. Der var simpelthen for mange mennesker om processen. Tidligere havde skolen arbejdet på en sådan måde, at der var \'en person som var ansvarlig for at lægge skemaet, her var dog det modsatte problem, med at det var svært ene mand at lave et skema som tilgodeså alle lærerer samt elever.

Ud fra vores samtale med skemalæggeren, blev det klart at der er et ønske om at tilgodese mange behov, og et tydeligt ønske fra Kærbyskolen var at fokusere på det pædagogiske element i forhold til skemalægningen, et ønske som ikke har været muligt at opfylde ved hjælp af de manuelt udarbejde skemaer. Dette ønske om et skema med pædagogiske overvejelser, udforskes nærmere i afsnit \ref{paedagogisk_laering}.


\subsubsection{Politikere}
Politikerne har igennem lovet opstillet nogle krav til hvordan timerne i skolen skal fordeles mellem de forskellige fag. Man kunne ved hjælp af en fælles automatiseret løsning på alle landets folkeskoler fastsætte nogle fælles retningslinjer for hvordan skemaet skal lægges. Man kunne igennem disse retningslinjer sikre sig at landets skoler opfylder de krav som det forventes fra politikernes side. 

\subsubsection{Andre interessenter}
Vi har i denne interessentanalyse ikke taget højde for alle interessenter. Vi har blandt andet ikke beskrevet elever og deres forældre. Disse har selvfølgelig også en interesse i at skemaet lægges efter nogle bestemte kriterier. Vi har dog valgt at vi vil fokusere på skemalæggeren og dennes opgave, og hvordan vi kan optimere denne proces.

I den forbindelse har vi valgt at kigge på hvilke overvejelser en skemalægger gør sig i udarbejdningen af et skema, og hvilke mennesker som har indflydelse på denne proces. Vi har derfor i denne analyse fokuseret på interessenter som skemalæggeren har et direkte ansvar overfor, og beskrevet disse.


%Mangler et eller andet...
%Skrammel som måske kan sættes ind et andet sted..

%I stedet for at en person ville skulle påtage sig opgaven at få dette nye puslespil til at gå op, er det blot få ting der skal ændres i et program, og et nyt skema vil være klart med det samme. Altså er man ikke længere afhængig af at en person skal kunne få de mange klasser og lærere til at spille sammen, men at et program kan finde frem til løsningen langt hurtigere. 

%Når der udbydes valgfag i folkeskolen, foregår dette ofte ved at prioritere en række fag, hvorefter skemalæggeren efter bedste evne kan forsøge at finde lærere til at undervise i disse fag. Der bliver derfor lagt et tidspunkt ind i skemaet der blot hedder ``valgfag'', hvor eleverne så kan gå til deres respektive valgfag. Det er ikke sikkert at det er muligt at alle kommer på de valghold de havde som første prioritet, da skemaerne simpelthen kan blive umulige at få til at gå op. En automatiseret løsning vil dog ikke have disse samme begrænsninger. Den store forskel mellem et menneske og en computer i denne sammenhæng er nemlig, at mens en menneskelig skemalægger kan få skemaerne til at gå op med 10 forskellige klassetrin, kan en computer få det til at gå op med små variationer i skemaet til den enkelte elevtype (elevtype skal her forstås som en unik sammensætning af påkrævede og valgfrie fag). Der er derfor intet der stopper programmet fra at lægge nogle elevers valgfag mandag eftermiddag, mens de for andre elever i samme klasse først har valgfag torsdag morgen.

%Ved hjælp af en automatiseret løsning, som derfor løser udfordringerne ved denne personlige skemalægning, opnår skolen altså både en økonomisk fordel og et større fagligt udbud til eleverne.
	\subsection{Pædagogisk læring}
\label{paedagogisk_laering}
På \school nævnte skemalæggeren et ønske om et skema udarbejdet med pædagogiske overvejelser. Altså et skema som på bedste vis tilgodeser elevernes behov, for at opnå den højest mulige indlæring.

Da skemaet i år blev udarbejdet manuelt, fandt Kærbyskolen det umuligt at lave et skema der tilgodeså alle de pædagogiske overvejelser de havde gjort sig, for samtlige klasser\cite{interview_Kaerby}.

På Tingstrup sikrede de sig at ingen klasser i indskolingen (0.-3. klassetrin) eller mellemtrinnet (4.-6. klassetrin) havde dage bestående udelukkende af kreative fag, eller dage bestående udelukkende af boglige fag. 

[Mere er i gang med at blive skrevet!]
	\subsection{State of the Art}
Indenfor skemalægning findes allerede nogle løsninger på de udfordringer, som nemt opstår indenfor skemalægning. Et af disse er USA Schedulers program kaldet School Master Scheduling Software\cite{USAS}, et program som tager de studerende i betragtning først og fremmest. Programmet er i stand til at lave funktionelle skemaer ud fra de studerenes ønsker, analysere de mest optimale løsninger ud fra ønskerne, og ud fra dette, undgår konflikter så som 2 moduler, der ligger oveni hinanden så studerende, der deltager aktivt i begge, bliver nødt til at vælge den ene fra.
Et andet stykke software kaldet Mimosa Scheduling Software\cite{Mimosa}, hvilket fokuserer mere på de grundlæggende udfordringer så som overbooking og nemt at kunne ændre på skemaerne, hvis der skulle komme en uventet udfordring. Det er i stand til at lave skemaer for lærere, elever, klasser og lokaler, alt det som enhver dansk folkeskole har brug for.
	\subsection{Brugerovervejelser}
\label{brugerovervejelser}
Da dette projekt handler om at udvikle et system til at hjælpe skolerne med deres skemalægning, nytter det ikke noget at lave et system, de ikke kan gennemskue og dermed bruge. Derfor er det relevant at kigge på brugernes færdigheder i forhold til at benytte informationssystemer.

På figur \ref{fig:docendo_skema} er skemalægningsprogrammet Docendo vist. Dette program blev tidligere brugt på Kærbyskolen, og som det ses af figuren, bruges drag\&drop-princippet til at indsætte moduler i skemaet. Dette gør programmet meget intuitivt og let at bruge - noget som Kærbyskolen efterspørger. I interviewet med Kærbyskolen fortalte skemalæggeren, at programmet skulle være meget let at bruge for, at det var en god løsning for skolen.

\begin{figure}[h!]
	\centering
	\includegraphics[width=\textwidth]{Docendo}
	\captionsource{Docendos skemalægningsprogram}{\url{https://docendo.dk/folkeskole.html}}
	\label{fig:docendo_skema}
\end{figure}

På skoler er der flere forskellige systemer der skal arbejde sammen for, at alt fungerer. Skemaet skal eksempelvis lægges ind på nettet på SkoleIntra, så eleverne kan få adgang til det. Derfor er det vigtigt, at et eventuelt program, der laves til at hjælpe med denne skemalægning, er kompatibelt med disse allerede eksisterende programmer, så der ikke skal manuelt arbejde til at konvertere filer fra det ene system til det andet. Heldigvis bruges der i de fleste programmer CSV-filer, hvilket er tekstfiler med kommaseparerede værdier, som så kan læses af de respektive programmer. Det der gør denne filtype nem at arbejde med, er, at vi ved at udskrive informationer omkring skemaet i forskellige rækkefølger, men i samme filformat, kan eksportere til alle disse programmer.

Ved at lave en måde at eksportere på til de forskellige formater, kan skemaet tages direkte fra vores program, og sættes ind i de andre programmer. Dette betyder for skolen, at programmet potentielt kun skal bruge informationer omkring skolen og lærere, og så kan den selv lave et skema, der er klar til, at de kan bruge det direkte.

Interviewet med Tingstrup skole fortalte, at det ikke var nødvendigt med en decideret grafisk brugerflade, så længe at der var noget som fortalte brugeren, hvad programmet lavede. Det vil sige, at det ikke er nødvendigt at have flot grafik, så længe brugeren kan se programmets status. 

Dermed er kravet fra de to skoler et system, som ikke behøves at have en grafisk brugerflade, men skal være nemt at bruge, samtidig med at det hele tiden skal oplyse om, hvor langt det er, så brugeren (skemalæggeren) kan se, om systemet er gået i stå, eller om der er opstået problemer. Når programmet skal laves, er det altså vigtigt, det holdes for øje, at brugerne af systemet ikke nødvendigvis ved hvad de skal, og at programmet derfor skal guide dem igennem processen med at indtaste data.
	\subsection{SWOT-analyse}
For at forstå \school bedre, kan vi lave en SWOT-analyse af skolen som en virksomhed. På denne måde ønsker vi at forstå mere om skolens virkemåde, samt se hvor vi med et program kan forbedre skolens muligheder.

\subsubsection*{Styrker}
I form af Kærbyskolens mindre størrelse på kun 335 elever og 33 lærere \cite{Kaerbyskolens-laerere}, er der muligheder for et tættere arbejde mellem de ansatte. Dette kom eksempelvis til udtryk, under skemalægningen hvor alle lærere og pædagoger arbejde sammen om at lægge et skema.

Selvom det ifølge skolen selv ikke var den største succes at lægge skemaet på denne måde, viser det alligevel hvilket tæt samarbejde de ansatte kan have med hinanden på skolen.

For skolen giver dette tætte samarbejde en højere kvalitet, da man ved at snakke med alle parter, sikrer sig at alle behov på skolen bliver hørt. Hvis vi igen tager udgangspunkt i den fælles skemalægning, betød samarbejdet at alle lærere og dermed alle klassetrin blev repræsenteret under skemalægningen, og at skemaerne derfor blev gode for samtlige klasser på skolen. 

\subsubsection*{Svagheder}
    -> Vikar 

\subsubsection*{Muligheder}
	-> 

\subsubsection*{Trusler}
Kærbyskolen har ikke lokaler nok på skolen, til alle dens fag. Det vil sige at lokaler til fag som idræt, musik og sløjd bliver lånt andetsteds fra. Skolen låner lokaler fra institutionen Kulturskolen, og skolen opnår dermed faciliteter til at undervise i disse fag. Denne afhængighed af Kulturskolen skaber dog nogle udfordringer og trusler mod Kærbyskolen. Lokalerne på kulturskolen er ikke altid ledige, og når skemaerne skal lægges, skal der tages højde for dette. 

Måden dette blev gjort på var ved at sætte nogle faste intervaller hvor hver klasse havde mulighed for at bruge lokalerne. Dette satte nogle store begrænsninger i forhold til skemalægninger, hvor man nu mistede megen frihed i forhold til placeringen af timerne, på grund af kravene til at låne lokaler. 
	\section{Problemafgrænsning}
	
I vores problemafgrænsning ønsker vi at indsnævre det problem, som tages med fra problemanalysen og finde frem til de fokusområder, som vi finder vigtigst inden for skemalægning i folkeskoler og ønsker ar tage udgangspunkt i under problemløsningen. Prolemafgrænsning skulle altså gerne hjælpe med at finde frem til en god problemformulering. Det vil i afsnittet altså blive beskrevet hvilke fokusområder der til- og fravælges.

For at sikre et projekt, som har relevans i vores tid og for vores specifikke case, tages der udgangspunkt i vores case, Kærbyskolen.

\subsection*{Pædagogisk læring}
På Kærbyskolen har skemaer, som tager deres udgangspunkt i pædagogisk læring og elevtrivsel i år været højt prioriteret. Dette kom til udtryk under interview med skolen, hvor det blev beskrevet, at pædagogiske overvejelser i forhold til skemalægning blev diskuteret meget og var noget, som nuværende løsninger i følge skolen ikke forholder sig nok til. Vi ønsker derfor at lægge fokus på skemalægning med vægt på pædagogisk overvejelser.

\subsubsection*{Brugergrænseflade}
Brugergrænseflade fravælges som fokusområde og dette gøres hovedsageligt grundet projektets omfang og tidsbegrænsning. Dette beytder dog ikke, at brugervenlighed glemmes helt. Kærbyskolen lagde under det foretagede interview vægt på, at programmer skal være nemme at bruge, og at det er vigtigt, at programmet løbende giver brugeren en status. Dette ønsker vi så at tage hensyn til i så stor et omfang, som det er muligt.

\subsubsection*{Andre afgrænsninger}
I forbindelse med overvejelser omkring projektets omfang og de opstillede tidsbegrænsninger, som vi må forholde os til, har vi desuden måtte gøre os nogle andre fravalg af fokuspunkter. Dette er valg som vi føler er nødvendige for, at projektet realistisk kan gennemføres i tide og samtidig ikke rammer den egentlige kvalitet af problemløsningen eller programfunktionalitet, men blot gør det endelige program lidt mindre.

Vi har med valget af Kærbyskolen som case fravalgt at tage højde for valgfag, da skolen kun går til 6. klasse, hvor disse endnu ikke findes. 

Vores case, Kærbyskolen, er en meget lille skole og derfor  har de ikke mulighed for at have alle fagene på skolensgrund. Kærbyskolen er nødt til at arbejde sammen med andre parter for at opfylde kraverne om idræts og hjemkundsskabsfaciliteter. Vi har dog valgt, at fokuset ikke skal være lagt på lokalerne. Men på pædagogisk læring og brugerflade.

Vi fokuserer på skemalægning på Kærbyskolen, men ønsker ikke at tage højde for skolens såkaldte J-klasser for autismediagnostiserede elever, da der dannes skemaer efter elevens individuelle behov. Derfor vil vi fokusere på at lægge skema til klasserne 0 til 6.

%\subsubsection*{Valgfag}
%Vi har fravalgt valgfag, som et fokusemne, da vi har valgt at arbejde ud fra en case, hvor skolen i dette tilfælde går op til 6. klasse, hvor de ikke har valgfag.

%\subsubsection*{Lokaler}
%Vores case, Kærbyskolen, er en meget lille skole og derfor  har de ikke mulighed for at have alle fagene på skolensgrund. Kærbyskolen er nødt til at arbejde sammen med andre parter for at opfylde kraverne om idræts og hjemkundsskabs faciliteter. Vi har dog valgt, at fokuset ikke skal være lagt på lokalerne, men på pædagogisk læring og brugerflade.

%Vi fokuserer på skemalægning på Kærbyskolen, men ønsker ikke at tage højde for skolens såkaldte J-klasser for autismediagnostiserede elever, da der dannes skemaer efter elevens individuelle behov. Derfor vil vi fokusere på at lægge skema til klasserne 0 til 6.

    \section{Problemformulering}
    På baggrund af de observationer og erfaringer, der er blevet gjort gennem rapportens problemanalyse, er problemet i afsnit \ref{afg} blevet afgrænset, og vi er kommet nærmere det, som vi mener er kernen i vores problemstilling om skemalægning i danske folkeskoler. Vi er nået frem til, at det der er behov for på folkeskoler ikke blot er en nemmere, hurtigere form for skemalægning, men også en skemalægning, som har fokus på pædagogiske overvejelser. Dette har givet os nedenstående problemformulering.

``Hvordan hjælper man, ved hjælp af software, skemalæggeren med at lægge vægt på pædagogiske overvejelser under skemalægingen?''
    \newpage
    \section{Problemløsning}
    \subsection{Løsningsstrategi}
Vi vil i vores projekt tage en anden tilgang end de produkter vi undersøgte i afsnit \ref{sota}. Vi vil i stedet gøre som naturen gør, og bruge evolution til at udvikle vores skemaer, noget som genetiske algoritmer er optimale til.

Opbygningen af programmet bliver derfor følgende: Vi har en genpulje hvori ``DNA'et'' for hver enkelt skema ligger. Vi vil så hele tiden forsøge at udvikle de bedste i vore genpulje til at blive bedre, mens resten uddør. Ved at ``dræbe'' de dårlige skemaer, sikrer vi os at vi hele tiden har en genpulje af en fast størrelse. Måden vi vil udvikle de bedste på, er ved at lave de dårlige DNA-strenge om til de gode DNA-strenge, dog med få ændringer, mens vi lader de gode DNA-strenge overleve. Flowet i programmet er illustreret på figur \ref{fig:loesningsdiagram}.

Vi sikrer os på denne måde at vi aldrig går baglæns i evolutionen, men kun forbedrer skemaer, for hvis et af de muterede skemaer viser sig at være bedre end de bedste nuværende DNA-strenge, vil dette muterede skemaer overtage pladsen som nummer 1 i genpuljen, og dermed sprede sig meget hurtigt til resten.

Hvis et nyt skema overtager pladsen som nummer 1, vil spredningen af denne foregå eksponentielt, da hvert skema vil sprede sig til to nye, indtil alle skemaerne er variationer af skemaet.

Læseren kan muligvis se problemet med denne metode. Ved at vi kun laver små ændringer i de gode skemaer, kan vi finde et lokalt maksimumspunkt inden for vurderingen af skemaet. Vi har altså nået toppen af vores `races' udvikling, uden nødvendigvis at have fundet det bedste skema. Måden vi vil komme uden om dette problem, er ved igen at se på hvordan naturen gør dette.

Alt liv på jorden stammer fra de mindste celler, som ved hjælp af mutation udviklede sig til hvordan livet ser ud i dag. Vores skemaer vil derfor også have en lille chance opleve en fuldstændig mutation, og dermed potentielt kunne forbedre hele races, med dens nye karakteristika.

\begin{figure}[h!]
	\centering
	\includegraphics[width=0.8\textwidth]{loesningsdiagram.png}
	\caption{Løsningsdiagram til vores program}
	\label{fig:loesningsdiagram}
\end{figure}
    \newpage
    
	\bibliographystyle{apa}
	\bibliography{sources}

    \section{Bilag}
    \begin{enumerate}
	\item Hvad er jeres nuværende procedure angående skemalægning?
	
	[text]
	\item Hvor lang tid bruger I på at lave skemaer? (før og efter reform?)
	
	[text]
	\item Bruger skolen faste skemaer eller ugentlige?
	
	[text]
	\item Hvor tit skal der ændres i skemaer?
	
	[text]
	\item Findes buffer-timer? 
	
	[text]
	\item Hvordan håndterer man timer der ikke finder sted?
	
	[text]
	\item Er lektiecafe på skemaet? Skal der være en lærer til stede (en lærer pr klasse)?
	
	[text]
	\item Er der faste lærer til hver klasse? Hvor længe beholder man samme lærer?
	
	[text]
	\item Hvordan håndterer i lokaler? (f.eks. ved store klasser?, hvad hvis 2 skal være i fysik-lokalet? Faste lokaler?)
	
	[text]
	\item Hvordan håndterer I lærernes tid på skolen? (i form af forberedelsestid)
	
	[text]
	\item Hvor mange timer om året pr lærer? pr elev? (Har skolen data?)
	
	[text]
	\item Hvilket system bruger I til håndtering eller udarbejdning af skemaer?
	
	[text]
	\item Hvordan samarbejder det med andre systemer? Kan du importere andre filer?
	
	[text]
	\item Fremvisning af programmet?
	
	[text]
	\item Vikarhåndtering, bliver det sat ind i systemet?
	
	[text]
	\item Bliver en vikartime regnet som en normal time? (uddannelsesniveauet er jo ikke det samme)
	
	[text]
	\item Komfortabel med at arbejde i CLI?
	
	[text]
	\item Prioritering af fag efter tid på dagen?
	
	[text]
	\item Prioritering af læreres ønsker?
	
	[text]
\end{enumerate}
    \begin{enumerate}
	\item Hvad er jeres nuværende procedure angående skemalægning?
	
	1 Viseskoleleder, 1 afdelingsleder og leder ligger grundskemaet. Viseskolederen laver finpudsningen.
	grundskemaet: Det, som gælder fra skoleårets start.
	\item Hvor lang tid bruger I på at lave skemaer? (før og efter reform?)
	
	Efter: ca. 8 arbejdsdage for alle ledere. ca. 3 dage på seminar og ca. 3 dage efter eleverne har fået ferie. Ved ikke om de hjælper hinanden eller om de bare kommer med inputs. Starter klokken 8, nogle gange først hjemme klokken ca. 22:00.
	\item Bruger skolen faste skemaer eller ugentlige?
	
	Alle elever har 2 skemaer. Et som kører fra sommerferie til jul, og et som kører fra jul til sommerferien.
	\item Hvor tit skal der ændres i skemaer?
	
	Hvis der er noget som ikke fungerer, hvis en lærer er længerevarende syg, eller hvis en lærer finder et andet arbejde, bliver der ændret i skemaet.
	\item Findes buffer-timer? 
	
	Nej. Hvis en dansk-lærer er syg, kan en anden lærer godt komme ind, og så lave undervisningen om. For eksempel, dansk læreren er syg, og så kommer en matematiklærer og overtager timen, og laver dansktimen om til matematik.
	\item Hvordan håndterer man timer der ikke finder sted?
	
	Hvis en lærer ikke er tilstede, kommer der en vikar på. Dette kan både være lærer, eller vikar.
	\item Er lektiecafe på skemaet? Skal der være en lærer til stede (en lærer pr. klasse)?
	
	Efter 1. August blev lektiecaf\'een obligatorisk, så den er på skemaet. Nogle gange ligger den først på dagen, andre dage ligger den sidst på dagen.
	En lærer skal være til stede på lektiecaféen. Lektiecaféen hedder faglig fordybelse. 7, 8, og 9. består af 12 klasser, og 12 lærer, og dette ligger samtidig. Ud af de 12 lærer kan en engelsk, en dansk og en matematik, så de kan de forskellige ting som eleverne har brug for hjælp til, og så må eleverne selv vælge hvilken faglig fordybelses time de går til. Har de brug for hjælp til Engelsk, kan de gå til Engelsk timen den ene dag, og den næste matematik, hvis dette er tilfældet.
	
	Der findes et stillerum, hvor eleverne kan gå hen og skrive hvad de skal i fred og ro. Der er en lærer til stede, som sørger for der er stille i dette rum.
	Det fungere dog kun sådan i overbygningen. I indskolingen og mellemklassen har hver lærer, hver deres klasse, som klassen skal være ved. De vælger ikke selv.
	Det hedder faglig fordybelse i alle klasser bortset fra 9. klasse, hvor det hedder lektiecafé. Dette kan muligvis skyldes at de selv vælger, hvilken ``café'' de går til.
	\item Er der faste lærer til hver klasse? Hvor længe beholder man samme lærer?
	
	Faste lærere.
	0 klasse kører for sig selv.
	Nye lærer fra 1. klasse, som fortsætter gennem hele indskolingen 1-3.
	Nye lærere igen fra 4-6 klasse, også klasselæreren. De flytter også klasselokalet.
	Nye lærere igen fra 7-9 klasse, også klasselæreren. De flytter også klasselokalet. I 7. klasse bliver der rykket sammen fra flere skoler, og der bliver dermed lavet nye klasser. De flytter klasselokaler igen i 9. klasse, da der er et området som er blevet renoveret, og som kun er tilegnet 9. klasser. Her sker der også af og til at klasserne har tværfaglige forløb, hvor to klasser går sammen og arbejder.
	\item Hvordan håndterer i lokaler? (f.eks. ved store klasser?, hvad hvis 2 skal være i fysik-lokalet? Faste lokaler?)
	
	I idræt har hele årgangen idræt samtidig. Ellers sørges der for at de andre lokaler sådan som madkundskab- og fysik-lokalerne, kun bliver lagt til rådighed for 1 klasse i et bestemt modul.
	\item Hvordan håndterer I lærernes tid på skolen? (i form af forberedelsestid)
	
	Lærernes på fuld tid har mødetid 7:30, og nogle dage har de længere dage end andre. Nogle dage møder de 7:30, og har fri klokken 17, da det er her, de har mødetid. Eleverne har fri klokken 15, så her er der møde fra 15:00 til 17:00. Altid møde om Tirsdagen, da det er her, lærerne er på skolen til 17. Lærerne arbejder ikke hele dagen, og den resterende tid, bruger de på ``andet'', hvilket bliver brugt til andet så som forberedelse, ringe til forældre, eller møder.
	\item Hvor mange timer om året pr. lærer? pr. elev? (Har skolen data?)
	
	Lærerne arbejder ca. 40 timer om ugen, og de arbejder 42 uger om året.
	
	0-3 går i skole fra 8 til 14:00 hver dag
	
	4-6 går i skole fra 8 til 14:20 hver dag.
	
	7-9 går i skole fra 8 til 15:05 hver dag.
	\item Hvilket system bruger I til håndtering eller udarbejdning af skemaer?
	
	De bruger KMD Educa Personale, kaldet Puls før sommerferien. Skemaprogrammet de bruger til at lave skemaerne i, hedder tabulex, et program de endnu ikke har brugt. Det program de brugte tidligere, hed Matrix.
	Skemaerne bliver lagt over i KMD Educa Personale, hvor der også bliver lavet vikardækning og hvor der bliver tjekket efter om de har mødetid anderledes. Hvis en lærer har haft klassemøde til klokken 20:00, har de arbejdet 3 timer for meget, og de kan derfor gå ind og ændre i skemaet så de møder senere eller går hjem tidligere, hvis de ikke har undervisning, og kun hvis de syntes de har udført deres arbejde. Lærerne kan derfor selv håndtere deres flex-tid. De kan også se hvor meget de har at bruge af.
	\item Hvordan samarbejder det med andre systemer? Kan du importere andre filer?
	
	Skemaerne importeres ind i KMD Educa Personale når det er færdigt fra Tabulex / Matrix, og fra kan der trykkes på en knap, som så bliver ført over til Intra, hvor alle andre kan se det. Smårettelser bliver lavet i Educa Personale.
	KMD Educa Personale har selv et skemalægningsprogram, som dog ikke er helt færdig endnu, og for svært at lægge skemaer i.
	\item Vikarhåndtering, bliver det sat ind i systemet?
	
	Vikardækning laves i Educa Personale, og så bliver det eksporteret til Intra, så lærere og vikarer kan se det.
	\item Bliver en vikartime regnet som en normal time? (uddannelsesniveauet er jo ikke det samme)
	
	En vikar får højere løn, da en læreres løn allerede ligger i deres løn, hvor en vikar skal forberede sig ud over.
	\item Komfortabel med at arbejde i CLI?
	
	Der skal være en grafisk brugerflade. Personligt er hun ligeglad med om det er en CLI eller en GUI, så længe man kan se der sker noget med det samme.
	\item Prioritering af fag efter tid på dagen?
	
	Det kan ikke lade sig gøre, så der bliver ikke prioriteret på sådan noget som fysik og idræt.
	Dog bliver der sørget for der ikke kun ligger kreative fag på samme dag (gælder dog ikke for overbygningen, da de har disse fag som valgfag), ligesom der heller ikke kun ligger de boglige på en dag.
	\item Prioritering af læreres ønsker?
	
	Lærerne laver en ønskeliste inden skemalægningen, hvor de blandt andet kan ønske efter at undervise i bestemte klassetrin og hvilke tidspunkter de helst vil undervise. Lærerne bestemmer fuldstændig selv, hvad de skriver på denne ønskeliste, og så kan skemalæggerne forsøge at gøre deres bedste på at opfylde så mange ønsker som muligt
\end{enumerate}
    \subsection{Interview med l�rer}
\label{InterviewLaerer}
\begin{enumerate}
	\item Hvad er din rolle?
    
    L�rer, der har v�ret med til at planl�gge skema.
   
   
    \item Hvilket team underviser du p�?
	    
	3-4 klasse (math).
    
    
    \item Hvor l�nge har personen arbejdet som l�rer?
	    
	11�r - barsel osv.
    
    
    \item Hvilke p�dagogiske overvejelser mener du at man burde g�re sig under skemal�gningen?
		
	Mange forskellige hensyn, Dansk samtidigt med paralleklasse, og forberedelse sammentid, s� de kan snakke om hvad de g�r osv. 
	
	Bev�gelsesb�nd, f�r eller efter frikvarteret, 8-10 i stedet for 8:00 - 9:30. L�seb�nd, s� der ligger timer sammen for en hel �rgang, (s� begge klasser f.eks. har l�seb�nd sammen).
	
	Ville v�re tr�t af mate efter middag, da eleverne tit er flade efter middag, dr�bvis barsel, (fri hver torsdag), om man kan indarbejde bindinger for l�rere osv. 
		
	Kulturskole, hvorn�r elever m� g� til eller fra kulturskolen. (m�ske snak med kulturskolen, da der er mange problematikker).
	
	Hvorn�r man gerne vil have fri eller time med en bestemt kollega, s� man kunne have hold sammen hvis det var. l�ringssamtaler. (skolehjemsamtale 14-17 onsdag). Alle skolens l�rer skal ikke have undervisning efter 14.. 
		
    
    \item Hvordan �nsker du at tiden p� skolen bruges (hvorn�r forberedelse og hvorn�r undervisning)
    
    Halv times forberedelse er tr�ls, da man lige skal igang igen. s� ikke for mange sm� huller. 
	
	Al forberedelse ogs� tr�ls at have p� kun 1-2 dage (da man kan misse det, hvis det er)
	
	Forberedelse ogs� meget dejligt at have om morgenen, da man godt kan v�re tr�t sent p� dagen.
    
    
    \item Hvor meget forberedelses tid f�ler du er n�dvendigt?
	
	Et godt udgangspunkt for forberedelestid, ville v�re en time af gange, dog er en halv time af og til fint nok af og til. En halvtimes forberedelsestid er dog tr�ls s� ofte.


    \item Er du tilfreds med dit nuv�rende skema?
	
	Processen tog meget lang tid, og mere hvad kan lade sig g�re, end hvad der er godt. Stor binding af sv�mning og idr�t. Sm� ting mand ender med at pusle med.
	
	Tog nok 2-3 dage, sv�rt at tage hensyn til �nsker/overvejelser. dvs. valgte at prioritere at det kunne lade sig g�re. Meget mentalt udmattende.
	
	Skemaet blev udem�rket.
    
    
    \item Hvor mange klasser er du l�rer for?
	
	2: (4. A og b) sidste �r for 4 forskellige klasser. 
    
    
    \item Hvordan mener du at skemal�gningen kan forbedres?
	
	At man kan v�lge mellem 2 muligheder. Eller at man har lidt forskellige muligheder generelt, da det vil hj�lpe med at passe flere p� en gang. 
	
	Har brugt matrix. 
    
    
    \item Hvis vi laver systemet, hvilke parametre synes I s� at vi skal tage h�jde for?
	
	UU underst�ttendeundervisning. Lektiecafe, og undervisning med p�dagoger.
	
	Har lige nu lektiecafe sidst p� dagen.  
    
    
    \item Eventuel kontaktoplysninger
	
	lottebirkholm@gmail.com
\end{enumerate}

    \newpage
    \input{bilag/timetabel.tex}
\end{document}