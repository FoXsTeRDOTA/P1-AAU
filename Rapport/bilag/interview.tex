\subsection{Interview med Tingstrup Skole}
\begin{enumerate}
	\item Hvad er jeres nuværende procedure angående skemalægning?

	Som regelt bølger det op og ned mellem teams og ledelsen, i år er det alle teams der har arbejdet sammen. De startede med at snakke om hvilke områder de ville prioritere, og så arbejdede de ud fra disse områder. De forsøgte i år at tage udgangspunkt i pædagogiske principper, dog har det været meget svært at gøre, grundet begrænsninger som f.eks. lokalemangel. I år har de inddelt lærerne i teams for de forskellige klassetrin. 0 klasse har sit eget team, og ellers er teamsene: 1-2 årgang, 3-4 og 5-6 årgang. Lærerne har så kun undervist i de klasser deres team håndtere (med meget få undtagelser).

	Ud fra deres oplevelser med skemalægningen i år har de fundet ud af, at ikke alle lærere skal være med til at lægge skemaet, samt at det er svært at indarbejde pædagogiske overvejelser og værdier i skemaet. Næste år vil de udpege ledere for hvert team, og så skal lederne arbejde sammen om at lave skemaet, så der bliver færre mennesker involveret.

	Selvom det var svært at ligge skemaerne i år, har måden de gjorde det på ført til gode samtaler både internt mellem de ansatte, men også mellem lærere og elever, for at afdække alles ønsker.

	Måden de gjorde det på sidste år var, at lærere hver især skrev en række ønsker, og gav dem til \'en person som derefter skulle udarbejde skemaet på baggrund af alle disse ønsker. Det havde dog ført til at lærernes forberedelsestid blev spredt ud over hele dagen, hvilket nogle lærere havde været utilfredse med. F.eks. kunne deres skema lyde: 45 minutters undervisning, 45 minutters pause, 90 minutters undervisning og så 45 minutters forberedelse. Lærerne ønskede på baggrund af dette at de mindst skulle have 60 minutters forberedelse af gangen, da det gav den bedste mulighed for at forberede sig ordentligt til timerne.

	Nogle lærerer var ansat på skolen, på 2/3 af den normale arbejdstid, der skulle i skemaet tages højde for dette, på en sådan måde at deres timer ikke overlappede med deres arbejde på andre institutioner.

	
	\item Hvor lang tid bruger I på at lave skemaer? (før og efter reform?)

	I år tog det meget lang tid, da de var 60 ansatte om det. Der var på forhånd bestemt nogle få ting, såsom at 1. klasse skulle have billedkunst om mandagen, men alle de små detaljer og præcise tidspunkter skulle der opnås enighed om. Planen var oprindeligt at det skulle tage 3-4 timer at nå frem til det skema, reelt tog det dog omkring 8 timer, for de 60 mennesker. 

	Tanken var også oprindeligt at der i løbet af et år skulle være 4 skemaer. Grundet at det tog så lang tid at lægge skemaet, og at det var et meget stort arbejde, er der ingen der har taget initiativ til at der skulle lægges et nyt, da der var gået 1/4 af skoleåret. Derfor bliver der enten kun 2 skemaer i løbet af året, eller i værste fald kun dette ene.

	Skolen har eksperimenteret med fagdage, altså hvor en hel dag var sat af til et enkelt fag. Erfaringerne fra dette var dog at det førte til meget spildtid, og at lærerene blev meget frustreret over det.

	Til at holde styr på timeantal for lærere og elever, har de tidligere brugt Microsoft Excel.

	
	\item Bruger skolen faste skemaer eller ugentlige?

	Skolen har som regel 4 skemaer på et år, hvor hver skoledag er inddelt i moduler på 30 minutters længde.

	\item Hvor tit skal der ændres i skemaer?

	Hvis der er noget i skemaet der ikke fungere, bliver der lavet små ændringer for at få det til at fungere, men ellers kun de 4 skemaer om året (som dog ikke er blevet gjort i år).


	\item Findes buffer-timer? 

	Der findes på skolen ingen ``buffer-timer''. Dette betyder at der ikke må ske aflysninger af timer, men at der i stedet skal sættes en vikar på.


	\item Er lektiecafe på skemaet? Skal der være en lærer til stede (en lærer pr klasse)?

	Der er lektiecaf\'e på skemaet. Der er lærere til hver klasse. Det vil sige at lektiecaf\'en virker helt som et normalt fag. Dog kan der godt være flere lærere til et fag, så de kan have lærere der dækker hvert fagområde. Lektecaf\'en har de lagt sidst på dagen, da dette fungerer bedst i forhold til hvilke lokaler der er til rådighed på kulturskolen, hvor de låner lokaler.

	
	\item Er der faste lærer til hver klasse? Hvor længe beholder man samme lærer?

	Ja. Lærene underviser i deres årgangs teams. dvs de underviser for 2 årgange om året. altså hvis man var på 1-2 klasses teamet underviste man 1-2 klasse.
	
	\item Hvordan håndterer i lokaler? (f.eks. ved store klasser?, hvad hvis 2 skal være i fysik-lokalet? Faste lokaler?)
	
	Bruger flex skema. Binder af lokaler pga. kulturskole
	
	\item Hvordan håndterer I lærernes tid på skolen? (i form af forberedelsestid)
	
	Et skema skal kunne understøtte læring og trivsel, eller sammenhængende blokke, så man vil kunne nå at fordybe sig. Det ville være dejligt at kunne tage højde for små ting i skemalægningen, frem for at sige om hver ting at enten vægter det, eller også gør det ikke. Altså kunne det have været rart at kunne sige hvor meget hver enkelt parameter betyder for skemaet. 
	
	\item Hvor mange timer om året pr lærer? pr elev? (Har skolen data?)
	
	Vil kunne måske få skoledata
	
	\item Hvilket system bruger I til håndtering eller udarbejdning af skemaer?

	I år brugte de håndkraft, klippeklister osv \ldots De plejede dog at lave det med et program kaldet Docendo. 
	
	\item Hvordan samarbejder det med andre systemer? Kan du importere andre filer?
	
	Skal kunne arbejde med SkoleIntra (det kan det system de brugte før) Der er dog en ny platform på vej, minuddanelse. ville kunne være dejligt hvis det skal kunne arbejde sammen med det. Ville være oplagt hvis det kunne samarbejde med et regneark. Når de var rigtigt mange var det dejligt at de havde det visuelt overblik, altså en grafisk fremstilling af skemaet, især når de arbejder i teams
	
	\item Fremvisning af programmet?
	
	Kan ikke da de ikke har noget. Viste dog SkoleIntra siden som de bruger. så forældre, elever og lærer kan fx se skemaet online osv\ldots 
	
	\item Vikarhåndtering, bliver det sat ind i systemet?
	
	Det bliver sat ind i skemaet. de er blevet sat ind i SkoleIntra, hvor man så ville kunne se det. Opgaverne er blevet givet til teamsne, hvor de har x timer til at være vikar. så hvis der er mangel på nogen så kan en anden tage over\ldots taget fra forberedelse, hvor de har lidt for mange timer i forvejen med det. Dog har de en presset hverdag, hvilket gør at de sætter ekstern vikar på. Meget vigtigt at kunne være komfortabel med. 
	
	\item Bliver en vikartime regnet som en normal time? (uddannelsesniveauet er jo ikke det samme)
	
	Ja.
	
	\item Komfortabel med at arbejde i CLI?
	
	Fortrækker noget meget nemt visuelt, drag\&drop, lige nu er de nødt lærerne er nødt til at komme ind til Jesper, hvorefter han så skal ændre det.
	
	\item Prioritering af fag efter tid på dagen?
	
	Der er intet fastsat fra ledelsen, men teamsne har snakket om det. Der har været snak om at mindske sværhedsgraden i de sene timer (14-15), hvor eleverne er ved at være trætte. De har haft mange diskussioner om hvordan det er bedst at håndtere den bevægelse som skal være i løbet af dagen. Der har været snak om at lave det som et fast modul fra 10 til 11, hver dag. Der har dog været en del bindinger i forhold til lokalerne, da det ikke er sikker at idrætslokalet er frit.
	

	\item Prioritering af læreres ønsker?

	Da skemaet blev lavet i teams af personalet, så lærene selv skulle tage holdning for hvad de ønskede at prioritere.  
	
\end{enumerate}