\begin{enumerate}
	\item Hvad er jeres nuværende procedure angående skemalægning?

	Helt team, anderledes i år. Normalt bølger det op og ned mellem teamet  og ledelsen. Så i År har teamene fået det. Så de har snakket hvad de ville prio og hvordan man så har kunne arbejde derfra. forsøgt at tage udgangspunkt i pedagoiske prænsipper, dog har det været meget svært, låst af nogen ting som fx. mangle på idræts faciliteter/ sløjt og hjemkunskab. I år forsøgt på noget andet.. i år valgt at sige... tilkunnet 2 årganeg 1-2 3-4 5-6 og så har et team for hver. inden for de klasser har du så også teams for fx 5 og 6. dog har lærene både haft timer for 5 og 6 klasse. De er blevet kloggere... ikke HELE teamet der skal ligge det. det var meget svært i forsøg med også at have pedagoiske karakteristiker og værdier, svært at lave efter det. Næste år vil de prøve det samme, dog med en form leder indenfor teamet, så lederne til sidst vil gå sammen om det. Måden de gjorde det i år har ikke virket så godt. Har givet en del reflektioner, der har ført til en masse snakken med eleverene og lærene. Før i tiden fik de massere af ønsker ind. kunne en person sætte det ind i et program, hvorefter de så vile forsøge. Sidste år stort problem med ikke sammensat forberedningstid. fx 45 undervisning, 45 pause, 90 time, 45 forberedelse fx. Vigtigt for dem er mere end 1 times forberedelse. Dog er det også vigitgt at det passer med børnene. 

De har også kombi ansatte, dvs. nogen der måske kun er 2/3 ansat på skolen, så de også vil skulle arbejde med noget andet, så de vil kunne   	
	
	\item Hvor lang tid bruger I på at lave skemaer? (før og efter reform?)
	
	Det har taget lang tid i år. De tog teamsne ind, hvorfra der i forvegen var delt lokaler ud til folk. fx 1 klasse har billedkunst mandag. Originalt tænkt 3-4 timer. brugt 3 gange så meget, hvor hele personalet var i gang. fx var der 8-9 igang. dvs de har cirka være 60 mand og cirka 8 timer. Og de fandt som sagt ud af at det ikke var alle der skullle have alle med i skemalægningen. Man kan finde ud af at noget ikke fungere. original tanke var 4 skema perioder. dog er det fx ikke blevet ændret. Fagdage har de også prøvet lidt med. fx dansk hver 3 tirsdag. De brugte for meget tid. og personalet blev meget fustreret over det. Dosendo? Dosindo? Dusindo? Dosindu? 

Har brugt regneark i forsøg på at beregne timemængde osv..

Største problem virker til at være 
Tror ikke man vil kunne tage højde for alt, så ville være fornuftingt hvis 
	
	\item Bruger skolen faste skemaer eller ugentlige?
	
	4 gange om året cirka
	kører haltimers moduler 

	\item Hvor tit skal der ændres i skemaer?


	\item Findes buffer-timer? 
	
	Ingen bufertimer. vikar... aldrig aflysninger
	

	\item Er lektiecafe på skemaet? Skal der være en lærer til stede (en lærer pr klasse)?

	Er der. Har de slåset med, pga kulturskole osv. det skal forældre give tilladelse til... lektiecafe lagt sidst på dagen pga binding fra kulturskolen.  

Lærer per klasse i dette. De er administreret som et fag , så den samme lærer på bestemte lærere. For 5-6 klasse 3-4 lærerer så man kan gå ind i et bestemt "fags" lokale. 
	
	\item Er der faste lærer til hver klasse? Hvor længe beholder man samme lærer?

	ja
	
	\item Hvordan håndterer i lokaler? (f.eks. ved store klasser?, hvad hvis 2 skal være i fysik-lokalet? Faste lokaler?)
	
	Bruger flex skema. Binder af lokaler pga. kulturskole
	
	\item Hvordan håndterer I lærernes tid på skolen? (i form af forberedelsestid)
	
	Et skema skal kunne underståtte læring of trivsel. eller sammehængende blokke, så man vil kunne nå at fordybe sig. Ville være dejligt at kunne have fordelt vægten til mere end bare vægter / vægter ikke. 
	
	\item Hvor mange timer om året pr lærer? pr elev? (Har skolen data?)
	
	
	
	\item Hvilket system bruger I til håndtering eller udarbejdning af skemaer?


	
	\item Hvordan samarbejder det med andre systemer? Kan du importere andre filer?
	
	Skal kunne arbejde med skoleintra ( kan det nu) Der er dog en ny platform på vej, minuddanelse. ville kunne være dejligt hvis det skal kunne arbejde sammen med et. Ville være oplagt hvis det kunne samarbejde med et regneark. Når de var rigtigt mange var det dejligt at de havde det visuelt. overblik overvejs er meget dejligt hvis det er.. især når de arbejder i teams
	
	\item Fremvisning af programmet?
	
	kan ikke
	
	\item Vikarhåndtering, bliver det sat ind i systemet?
	
	sat ind i skemaet. de er blevet sat ind i skoleintra, hvor man så ville kunne se det. opgaverne er blevet givet til teamsne, hvor de har x timer til at være vikar. så hvis der er mangle på nogen så kan en anden tage over... taget fra forberedelse, hvor de har lidt for mange timer i forvejen med det. Dog har de en præset hverdag, hvilket gør at de sætter ekstern vikar på. Meget vigtigt at kunne være komfertible med 
	
	\item Bliver en vikartime regnet som en normal time? (uddannelsesniveauet er jo ikke det samme)
	
	
	
	\item Komfortabel med at arbejde i CLI?
	
	Fortrækker noget MEGET nemt visuelt, dag&drop, lige nu er de nød til at kører med at lærene er nød til at komme ind til Jesper, hvorefter han så skal ændre det.  
	
	\item Prioritering af fag efter tid på dagen?
	
	Ikke nogen overvejelse fra ledelsen.. har været på teamsne .. har et bud på at de nok er bredt ud, så det ikke er. Der har også været snak om sværheden af undervisning ved fx 14-15. De har haft en del debater om bevægelse, om man skulle være firkantet, have det som et bond ( fx fra 10-11 )  så de lavet noget aktivt ændten før eller efter pause... Dette er blevet gjort på mange forskellige måder, da de som sagt har brugt en masse teams.. De har været låst med tidspunkter hvor de kunne have fx idræt pga. binding fra kulturskolen. Jo ældre de er, jo flere bindinger.  dvs. næste år vil de starte med 5-6 klasse osv. de har kunne snakke om brug af lokaler med andre teams. pedagoer mere med i skemaerne. men skal stadig have sammenhængdene dage, hvor pedagoerne sf. også skal være med i det fritids efter skolen. plotter selv skemaerne de har lavet ind i intra... hvorefter de så kan sætte skemabrikker på... man skal lave "brikker" for at kunne sætte et fag ind... dog virker det ikke til at man kan slette gamle brikker (Jesper viste ikke)
	programmet kan tage regneark Det er vist en del af it's learning. 

Syntes opgaven var vigtig for lærene, da det var dem der skulle kunne udholde skemaet i noget tid. Og man skal nok ikke starte på bar bund. 
	\item Prioritering af læreres ønsker?

	
	
\end{enumerate}