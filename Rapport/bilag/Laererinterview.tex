\subsection{Interview med lærer}
\label{InterviewLaerer}
\begin{enumerate}
	\item Hvad er din rolle?
    
    Lærer, der har været med til at planlægge skema.
   
   
    \item Hvilket team underviser du på?
	    
	3-4 klasse, som matematiklærer.
    
    
    \item Hvor længe har personen arbejdet som lærer?
	    
	11år, minus barsel og lignende.
    
    
    \item Hvilke pædagogiske overvejelser mener du at man burde gøre sig under skemalægningen?
		
	Mange forskellige hensyn, Dansk samtidigt med paralleklasse, og forberedelse sammentid, så de kan snakke om hvad de gør osv, eventuelt arbejde på tværs af klasserne.
	
	Bevægelsesbånd, før eller efter frikvarteret, 8:00 - 10:00 i stedet for 8:00 - 9:30. Læsebånd, så der ligger timer sammen for en hel årgang, så begge klasser f.eks. har læsebånd sammen.
	
	Ville være træt af mate efter middag, da eleverne tit er flade efter middag, dråbvis barsel, (fri hver torsdag), om man kan indarbejde bindinger for lærere osv. 
		
	Kulturskole, hvornår elever må gå til eller fra kulturskolen. Måske kan man snakke med kulturskolen, pga. de mange problematikker.
	
	Hvornår man gerne vil have fri eller time med en bestemt kollega, så man kunne have hold sammen hvis det var. Læringssamtaler. (skolehjemsamtale 14-17 onsdag). Alle skolens lærere skal ikke have undervisning efter 14.. 
		
    
    \item Hvordan ønsker du at tiden på skolen bruges (hvornår forberedelse og hvornår undervisning)
    
    Halv times forberedelse er træls og for kort, da man lige skal i gang igen. Så ikke for mange små huller. 
	
	Forberedelse kun fordelt 1-2 dage er også træls, da man kan være uheldig at misse de dage, hvis det er, eventuelt pga. sygdom.
	
	Forberedelse også meget dejligt at have om morgenen, da man godt kan være træt sent på dagen.
    
    
    \item Hvor meget forberedelses tid føler du er nødvendigt?
	
	Et godt udgangspunkt for forberedelestid, ville være en time af gangen, dog er en halv time af og til fint nok. En halvtimes forberedelsestid er dog træls for ofte.


    \item Er du tilfreds med dit nuværende skema?
	
	Processen tog meget lang tid, og mere hvad kan lade sig gøre, end hvad der er godt. Stor binding af svømning og idræt. Små ting mand ender med at pusle med.
	
	Tog nok 2-3 dage, svært at tage hensyn til ønsker/overvejelser. dvs. valgte at prioritere at det kunne lade sig gøre. Meget mentalt udmattende.
	
	Skemaet blev udemærket.
    
    
    \item Hvor mange klasser er du lærer for?
	
	2: (4. A og b) sidste år for 4 forskellige klasser. 
    
    
    \item Hvordan mener du at skemalægningen kan forbedres?
	
	At man kan vælge mellem 2 muligheder. Eller at man har lidt forskellige muligheder generelt, da det vil hjælpe med at passe flere på en gang. 
	
	Har brugt matrix. 
    
    
    \item Hvis vi laver systemet, hvilke parametre synes I så at vi skal tage højde for?
	
	UU understøttendeundervisning. Lektiecafe, og undervisning med pædagoger.
	
	Har lige nu lektiecafe sidst på dagen.  
    
    
    \item Eventuel kontaktoplysninger
	
	lottebirkholm@gmail.com
\end{enumerate}
