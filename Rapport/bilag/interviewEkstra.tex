\subsection{Interview med Tingstrup Skole}
\label{InterviewTingstrup}
\begin{enumerate}
	\item Hvad er jeres nuværende procedure angående skemalægning?
	
	1 Viseskoleleder, 1 afdelingsleder og leder ligger grundskemaet. Viseskolederen laver finpudsningen.
	grundskemaet: Det, som gælder fra skoleårets start.
	\item Hvor lang tid bruger I på at lave skemaer? (før og efter reform?)
	
	Efter: ca. 8 arbejdsdage for alle ledere. ca. 3 dage på seminar og ca. 3 dage efter eleverne har fået ferie. Ved ikke om de hjælper hinanden eller om de bare kommer med inputs. Starter klokken 8, nogle gange først hjemme klokken ca. 22:00.
	\item Bruger skolen faste skemaer eller ugentlige?
	
	Alle elever har 2 skemaer. Et som kører fra sommerferie til jul, og et som kører fra jul til sommerferien.
	\item Hvor tit skal der ændres i skemaer?
	
	Hvis der er noget som ikke fungerer, hvis en lærer er længerevarende syg, eller hvis en lærer finder et andet arbejde, bliver der ændret i skemaet.
	\item Findes buffer-timer? 
	
	Nej. Hvis en dansk-lærer er syg, kan en anden lærer godt komme ind, og så lave undervisningen om. For eksempel, dansk læreren er syg, og så kommer en matematiklærer og overtager timen, og laver dansktimen om til matematik.
	\item Hvordan håndterer man timer der ikke finder sted?
	
	Hvis en lærer ikke er tilstede, kommer der en vikar på. Dette kan både være lærer, eller vikar.
	\item Er lektiecafe på skemaet? Skal der være en lærer til stede (en lærer pr. klasse)?
	
	Efter 1. August blev lektiecaf\'een obligatorisk, så den er på skemaet. Nogle gange ligger den først på dagen, andre dage ligger den sidst på dagen.
	En lærer skal være til stede på lektiecaféen. Lektiecaféen hedder faglig fordybelse. 7, 8, og 9. består af 12 klasser, og 12 lærer, og dette ligger samtidig. Ud af de 12 lærer kan en engelsk, en dansk og en matematik, så de kan de forskellige ting som eleverne har brug for hjælp til, og så må eleverne selv vælge hvilken faglig fordybelses time de går til. Har de brug for hjælp til Engelsk, kan de gå til Engelsk timen den ene dag, og den næste matematik, hvis dette er tilfældet.
	
	Der findes et stillerum, hvor eleverne kan gå hen og skrive hvad de skal i fred og ro. Der er en lærer til stede, som sørger for der er stille i dette rum.
	Det fungere dog kun sådan i overbygningen. I indskolingen og mellemklassen har hver lærer, hver deres klasse, som klassen skal være ved. De vælger ikke selv.
	Det hedder faglig fordybelse i alle klasser bortset fra 9. klasse, hvor det hedder lektiecafé. Dette kan muligvis skyldes at de selv vælger, hvilken ``café'' de går til.
	\item Er der faste lærer til hver klasse? Hvor længe beholder man samme lærer?
	
	Faste lærere.
	0 klasse kører for sig selv.
	Nye lærer fra 1. klasse, som fortsætter gennem hele indskolingen 1-3.
	Nye lærere igen fra 4-6 klasse, også klasselæreren. De flytter også klasselokalet.
	Nye lærere igen fra 7-9 klasse, også klasselæreren. De flytter også klasselokalet. I 7. klasse bliver der rykket sammen fra flere skoler, og der bliver dermed lavet nye klasser. De flytter klasselokaler igen i 9. klasse, da der er et området som er blevet renoveret, og som kun er tilegnet 9. klasser. Her sker der også af og til at klasserne har tværfaglige forløb, hvor to klasser går sammen og arbejder.
	\item Hvordan håndterer i lokaler? (f.eks. ved store klasser?, hvad hvis 2 skal være i fysik-lokalet? Faste lokaler?)
	
	I idræt har hele årgangen idræt samtidig. Ellers sørges der for at de andre lokaler sådan som madkundskab- og fysik-lokalerne, kun bliver lagt til rådighed for 1 klasse i et bestemt modul.
	\item Hvordan håndterer I lærernes tid på skolen? (i form af forberedelsestid)
	
	Lærernes på fuld tid har mødetid 7:30, og nogle dage har de længere dage end andre. Nogle dage møder de 7:30, og har fri klokken 17, da det er her, de har mødetid. Eleverne har fri klokken 15, så her er der møde fra 15:00 til 17:00. Altid møde om Tirsdagen, da det er her, lærerne er på skolen til 17. Lærerne arbejder ikke hele dagen, og den resterende tid, bruger de på ``andet'', hvilket bliver brugt til andet så som forberedelse, ringe til forældre, eller møder.
	\item Hvor mange timer om året pr. lærer? pr. elev? (Har skolen data?)
	
	Lærerne arbejder ca. 40 timer om ugen, og de arbejder 42 uger om året.
	
	0-3 går i skole fra 8 til 14:00 hver dag
	
	4-6 går i skole fra 8 til 14:20 hver dag.
	
	7-9 går i skole fra 8 til 15:05 hver dag.
	\item Hvilket system bruger I til håndtering eller udarbejdning af skemaer?
	
	De bruger KMD Educa Personale, kaldet Puls før sommerferien. Skemaprogrammet de bruger til at lave skemaerne i, hedder tabulex, et program de endnu ikke har brugt. Det program de brugte tidligere, hed Matrix.
	Skemaerne bliver lagt over i KMD Educa Personale, hvor der også bliver lavet vikardækning og hvor der bliver tjekket efter om de har mødetid anderledes. Hvis en lærer har haft klassemøde til klokken 20:00, har de arbejdet 3 timer for meget, og de kan derfor gå ind og ændre i skemaet så de møder senere eller går hjem tidligere, hvis de ikke har undervisning, og kun hvis de syntes de har udført deres arbejde. Lærerne kan derfor selv håndtere deres flex-tid. De kan også se hvor meget de har at bruge af.
	\item Hvordan samarbejder det med andre systemer? Kan du importere andre filer?
	
	Skemaerne importeres ind i KMD Educa Personale når det er færdigt fra Tabulex / Matrix, og fra kan der trykkes på en knap, som så bliver ført over til Intra, hvor alle andre kan se det. Smårettelser bliver lavet i Educa Personale.
	KMD Educa Personale har selv et skemalægningsprogram, som dog ikke er helt færdig endnu, og for svært at lægge skemaer i.
	\item Vikarhåndtering, bliver det sat ind i systemet?
	
	Vikardækning laves i Educa Personale, og så bliver det eksporteret til Intra, så lærere og vikarer kan se det.
	\item Bliver en vikartime regnet som en normal time? (uddannelsesniveauet er jo ikke det samme)
	
	En vikar får højere løn, da en læreres løn allerede ligger i deres løn, hvor en vikar skal forberede sig ud over.
	\item Komfortabel med at arbejde i CLI?
	
	Der skal være en grafisk brugerflade. Personligt er hun ligeglad med om det er en CLI eller en GUI, så længe man kan se der sker noget med det samme.
	\item Prioritering af fag efter tid på dagen?
	
	Det kan ikke lade sig gøre, så der bliver ikke prioriteret på sådan noget som fysik og idræt.
	Dog bliver der sørget for der ikke kun ligger kreative fag på samme dag (gælder dog ikke for overbygningen, da de har disse fag som valgfag), ligesom der heller ikke kun ligger de boglige på en dag.
	\item Prioritering af læreres ønsker?
	
	Lærerne laver en ønskeliste inden skemalægningen, hvor de blandt andet kan ønske efter at undervise i bestemte klassetrin og hvilke tidspunkter de helst vil undervise. Lærerne bestemmer fuldstændig selv, hvad de skriver på denne ønskeliste, og så kan skemalæggerne forsøge at gøre deres bedste på at opfylde så mange ønsker som muligt
\end{enumerate}