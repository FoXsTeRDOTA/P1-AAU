	\documentclass[hidelinks, 12pt]{article}
\makeatletter
\newcommand{\group}{B2-2}
\newcommand\footnoteref[1]{\protected@xdef\@thefnmark{\ref{#1}}\@footnotemark}
\makeatother
\usepackage{graphicx}
\usepackage{graphicx,hyperref,amsmath,bm,url}
\usepackage[numbers]{natbib}
\usepackage{microtype,todonotes}
\usepackage{a4}
\usepackage[compact,small]{titlesec}
\usepackage[utf8]{inputenc}
\clubpenalty = 10000
\widowpenalty = 10000
\usepackage[T1]{fontenc}
\graphicspath{ {../Billeder/} }
\usepackage{tikz}
\usetikzlibrary{calc}
\usetikzlibrary{shapes}
\usepackage[labelfont=bf]{caption}
\renewcommand{\figurename}{\textbf{Figur}}
\renewcommand{\contentsname}{Indholdfortegnelse}
\renewcommand{\refname}{Referencer}

\makeatletter
\newdimen\@myBoxHeight%
\newdimen\@myBoxDepth%
\newdimen\@myBoxWidth%
\newdimen\@myBoxSize%
\newcommand{\SquareBox}[2][]{%
    \settoheight{\@myBoxHeight}{#2}% Record height of box
    \settodepth{\@myBoxDepth}{#2}% Record depth of box
    \settowidth{\@myBoxWidth}{#2}% Record width of box
    \pgfmathsetlength{\@myBoxSize}{max(\@myBoxWidth,(\@myBoxHeight+\@myBoxDepth))}%
    \tikz \node [shape=rectangle, shape aspect=1,draw=red,inner sep=2\pgflinewidth, minimum size=\@myBoxSize,#1] {#2};%
}%
\makeatother
\newcommand*{\captionsource}[2]{%
  \caption[{#1}]{%
    #1%
    \\\hspace{\linewidth}%
    \textbf{Kilde:} #2%
  }%
}
\newcommand{\tabitem}{~~\llap{\textbullet}~~}
\begin{document}
	
	\title{Projektledelse}
	\author{Gruppe \group}
	\date{\today \\\vspace{0.5cm} Version 1}
	\maketitle
	
	\section*{Værktøjer til tidsstyring og håndtering af opgaver}
	Vi har i gruppen valgt at lave en tidsplan hvori der indtastes deadlines og aftaler i gruppen, men også ting som afleveringer i andre kurser. På den måde ønsker vi at få et overblik over hvornår der er travlt i gruppen, og dermed få mulighed for at planlægge efter dette.

	I gruppen afslutter vi desuden hvert et gruppemøde med at lave en plan for det næste møde. På denne måde vil vi sikre os at vi ikke ender med at sidde til møder hvor vi `løber tør for opgaver'.

	Til kommunikation i gruppen benytter vi Slack (\url{http://www.slack.com}), hvor vi kan oprette forskellige kanaler til forskellige dele af projektet. På den måde kan vi opdele vores kommunikation for at gøre det overskueligt. 

	For at sikre os at alle ting bliver lavet (f.eks. at skrevne afsnit læses igennem og rettes) bruger vi Trello (\url{http://www.trello.com}), hvor vi har indsat de forskellige underafsnit, og kan flytte dem mellem forskellige stadier (ikke lavet/mangler rettelse/færdig). Vi kan desuden sætte mennesker på de forskellige opgaver, for at sikre os at det bliver lavet. Vi har i gruppen udnævnt en Trello Master (Mathias Jakobsen), som skal sørge for at Trello holdes opdateret, samt skal uddele mindre opgaver for at sørge for at alle mennesker bliver sat ind i alt stof i projektet.

	\section*{Grupperoller}
	Ud fra Belbins grupperoller har vi kunnet fastslå at gruppen er i mangel på en id\'emand. Dette betyder selvfølgeligt at vi skal være opmærksomme på at være fælles om at komme på nye id\'er. Vi skal derfor holde brainstorms sammen, så vi alle sammen kan hjælpe med at finde de id\'eer som ikke kommer til os af sig selv.

	\section*{Formelle roller i gruppen}
	I gruppen har vi følgende formelle roller:
	\begin{description}
		\item[Styringsgruppemedlem] - Mathias Jakobsen går til styringsgruppemøder og formidler gruppens fælles holdninger til disse.
		\item[Trello Master] - Mathias Jakobsen står for at holde Trello opdateret med nye punkter samt flytter punkter til deres nye positioner. Han skal desuden sørge for at tildele personer arbejdet til disse punkter.
		\item[Styrer af tidsplan] - Søren Madsen står for at holde tidsplanen opdateret med nye deadlines, og hvis noget i tidsplanen ikke kan lade sig gøre.
		\item[`Kommunikationschef'] - Søren Madsen skal være ansigtet udadtil. Det er denne person som tager kontakt til vejlederen, samt hvis gruppen skal kontakte nogen uden for projektet.
		\item[Ordstyrer og referent] - disse roller går på skift, og bestemmes dagen før et vejledermøder.
		\item[Håndtering af konflikter] - Theresa Walker står for at snakke med de mennesker som ender i konflikter. 
	\end{description}

\end{document}